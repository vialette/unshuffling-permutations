%%%%%%%%%%%%%%%%%%%%%%%%%%%%%%%%%%%%%%%%%%%%%%%%%%%%%%%%%%%%%%%%%%%%%%%%%%%%%%%

\documentclass[a4paper,10pt]{llncs}

%%%%%%%%%%%%%%%%%%%%%%%%%%%%%%%%%%%%%%%%%%%%%%%%%%%%%%%%%%%%%%%%%%%%%%%%%%%%%%%

\usepackage[utf8x]{inputenc}
\usepackage{amssymb}
\usepackage{amsmath}
\usepackage{amsfonts}
\usepackage{graphicx}
%\usepackage{llncsdoc}
\usepackage[english]{babel}
\usepackage[ruled,vlined,linesnumbered]{algorithm2e}
\usepackage{algorithmic}
\usepackage{float}
\usepackage{todonotes}
\usepackage{subfigure}
%\usepackage[mathcal]{euscript}
\usepackage{color}
\newcommand{\rt}{\textcolor{red}}
%%%%%%%%%%%%%%%%%%%%%%%%%%%%%%%%%%%%%%%%%%%%%%%%%%%%%%%%%%%%%%%%%%%%%%%%%%%%%%%

\input{matrix}
\input{math}
%%% 
%%% complexity.tex
%%% 

\usepackage{xspace}

%%% ----------------------------------------------------------------------
%%% complexity classes
%%% ----------------------------------------------------------------------

% TIME
\newcommand{\DTIMEX}{{\sf\bf DTIME}}
\newcommand{\DTIMEclass}{\DTIMEX\xspace}
\newcommand{\DTIME}{\DTIMEclass}
% NL class
\newcommand{\NLclassbase}{{\sf\bf NL}}
\newcommand{\NLclass}{\NLclassbase\xspace}
% P class
\newcommand{\Pclassbase}{{\sf\bf P}}
\newcommand{\Pclass}{\Pclassbase\xspace}
% NP class
\newcommand{\NPclassbase}{{\sf\bf NP}}
\newcommand{\NPclass}{\NPclassbase\xspace}
% coNP class
\newcommand{\coNPclassbase}{{\sf\bf coNP}}
\newcommand{\coNPclass}{\coNPclassbase\xspace}
% PSPACE class
\newcommand{\PSPACEclassbase}{{\sf\bf PSPACE}}
\newcommand{\PSPACEclass}{\PSPACEclassbase\xspace}
% MAXSNP class
\newcommand{\MaxSNPclassbase}{{\sf\bf MaxSNP}}
\newcommand{\MaxSNPclass}{\MaxSNPclassbase\xspace}
% MAXNP class
\newcommand{\MaxNPclassbase}{{\sf\bf MaxNP}}
\newcommand{\MaxNPclass}{\MaxNPclassbase\xspace}
% EPTAS class
\newcommand{\EPTASclassbase}{{\sf\bf EPTAS}}
\newcommand{\EPTASclass}{\EPTASclassbase\xspace}
% FPTAS class
\newcommand{\FPTASclassbase}{{\sf\bf FPTAS}}
\newcommand{\FPTASclass}{\FPTASclassbase\xspace}
% PTAS class
\newcommand{\PTASclassbase}{{\sf\bf PTAS}}
\newcommand{\PTASclass}{\PTASclassbase\xspace}
% APX class
\newcommand{\APXclassbase}{{\sf\bf APX}}
\newcommand{\APXclass}{\APXclassbase\xspace}
% log-APX class
\newcommand{\logAPXclassbase}{{\sf\bf log{\tt -}APX}}
\newcommand{\logAPXclass}{\logAPXclassbase\xspace}
% poly-APX class
\newcommand{\polyAPXclassbase}{{\sf\bf poly{\tt -}APX}}
\newcommand{\polyAPXclass}{\polyAPXclassbase\xspace}
% exp-APX class
\newcommand{\expAPXclassbase}{{\sf\bf exp{\tt -}APX}}
\newcommand{\expAPXclass}{\expAPXclassbase\xspace}
% NPO class
\newcommand{\NPOclassbase}{{\sf\bf NPO}}
\newcommand{\NPOclass}{\NPOclassbase\xspace}
% #P class
\newcommand{\sharpPclassbase}{\#{\sf\bf P}}
\newcommand{\sharpPclass}{\sharpPclassbase\xspace}
% FPT class
\newcommand{\FPTclassbase}{{\sf\bf FPT}}
\newcommand{\FPTclass}{\FPTclassbase\xspace}
% W class
\newcommand{\Wclassbase}[1]{{\sf\bf W[#1]}}
\newcommand{\Wclass}[1]{\Wclassbase{#1}\xspace}
% W class
\newcommand{\XPclassbase}{{\sf\bf XP}}
\newcommand{\XPclass}{\XPclassbase\xspace}
% WNL class
\newcommand{\WNLclassbase}{{\sf\bf WNL}}
\newcommand{\WNLclass}{\WNLclassbase\xspace}
% ZPP class
\newcommand{\ZPPclassbase}{{\sf\bf ZPP}}
\newcommand{\ZPPclass}{\ZPPclassbase\xspace}
% NPK class
\newcommand{\NPKclassbase}{{\sf\bf NPK}}
\newcommand{\NPKclass}{\NPKclassbase\xspace}
\newcommand{\NPKandclass}{\text{$\NPKclass_\text{and}$}\xspace}
\newcommand{\NPKzeroandclass}{\text{$\NPKclass^0_\text{and}$}\xspace}
\newcommand{\NPKorclass}{\text{$\NPKclass_\text{or}$}\xspace}
\newcommand{\NPKzeroorclass}{\text{$\NPKclass^0_\text{or}$}\xspace}

%%% ----------------------------------------------------------------------
%%% complete
%%% ----------------------------------------------------------------------

% keyword
\newcommand{\complete}{\text{-complete}}
% NL-complete
\newcommand{\NLcomplete}{\NLclassbase\complete\xspace}
\newcommand{\NLC}{\NLcomplete}
% P-complete
\newcommand{\Pcomplete}{\Pclassbase\complete\xspace}
\newcommand{\PC}{\Pcomplete}
% NP-complete
\newcommand{\NPcomplete}{\NPclassbase\complete\xspace}
\newcommand{\NPC}{\NPcomplete}
% coNP-complete
\newcommand{\coNPcomplete}{\coNPclassbase\complete\xspace}
\newcommand{\coNPC}{\coNPcomplete}
% PSPACE-complete
\newcommand{\PSPACEcomplete}{\PSPACEclassbase\complete\xspace}
\newcommand{\PSPACEC}{\PSPACEcomplete}
% MAXSNP-complete
\newcommand{\MaxSNPcomplete}{\MaxSNPclassbase\complete\xspace}
\newcommand{\MaxSNPC}{\MaxSNPcomplete}
% APX-complete
\newcommand{\APXcomplete}{\APXclassbase\complete\xspace}
\newcommand{\APXC}{\APXcomplete}
% #P-complete
\newcommand{\sharpPcomplete}{\sharpPclassbase\complete\xspace}
\newcommand{\sharpPC}{\sharpPcomplete}
% W[i]-complete
\newcommand{\Wcomplete}[1]{\Wclassbase{#1}\complete\xspace}
\newcommand{\WC}[1]{\Wcomplete{#1}}
% WNL-complete
\newcommand{\WNLcomplete}{\WNLclassbase\complete\xspace}
\newcommand{\WNLC}{\WNLcomplete}

%%% ----------------------------------------------------------------------
%%% hard
%%% ----------------------------------------------------------------------

% keyword
\newcommand{\hard}{\text{-hard}}
% NL-hard
\newcommand{\NLhard}{\NLclassbase\hard\xspace}
\newcommand{\NLH}{\NLhard}
% P-hard
\newcommand{\Phard}{\NPclassbase\hard\xspace}
\newcommand{\PH}{\Phard}
% NP-hard
\newcommand{\NPhard}{\NPclassbase\hard\xspace}
\newcommand{\NPH}{\NPhard}
% coNP-hard
\newcommand{\coNPhard}{\coNPclassbase\hard\xspace}
\newcommand{\coNPH}{\coNPhard}
% PSPACE-hard
\newcommand{\PSPACEhard}{\PSPACEclassbase\hard\xspace}
\newcommand{\PSPACEH}{\PSPACEhard}
% MAXSNP-hard
\newcommand{\MaxSNPhard}{\MaxSNPclassbase\hard\xspace}
\newcommand{\MaxSNPH}{\MaxSNPhard}
% APX-hard
\newcommand{\APXhard}{\APXclassbase\hard\xspace}
\newcommand{\APXH}{\APXhard}
% WNL-hard
\newcommand{\WNLhard}{\WNLclassbase\hard\xspace}
\newcommand{\WNLH}{\WNLhard}
% #P-hard
\newcommand{\sharpPhard}{\sharpPclassbase\hard\xspace}
\newcommand{\sharpPH}{\sharpPhard}
% W[i]-hard
\newcommand{\Whard}[1]{\Wclassbase{#1}\hard\xspace}
\newcommand{\WH}[1]{\Whard{#1}}

%%% ----------------------------------------------------------------------
%%% hardness
%%% ----------------------------------------------------------------------

% keyword
\newcommand{\hardness}{\text{-hardness}}
% NP-hardness
\newcommand{\NPhardness}{\NPclassbase\hardness\xspace}
% APX-hardness
\newcommand{\APXhardness}{\APXclassbase\hardness\xspace}
% W[i]-hardness
\newcommand{\Whardness}[1]{\Wclassbase{#1}\hardness\xspace}
% WNL-hardness
\newcommand{\WNLhardness}{\WNLclassbase\hardness\xspace}

%%% ----------------------------------------------------------------------
%%% completeness
%%% ----------------------------------------------------------------------

% keyword
\newcommand{\completeness}{\text{-completeness}}
% NL-completeness
\newcommand{\NLcompleteness}{\NLclassbase\completeness\xspace}
% P-completeness
\newcommand{\Pcompleteness}{\NPclassbase\completeness\xspace}
% NP-completeness
\newcommand{\NPcompleteness}{\NPclassbase\completeness\xspace}
% APX-completeness
\newcommand{\APXcompleteness}{\APXclassbase\completeness\xspace}
% #P-completeness
\newcommand{\sharpPcompleteness}{\sharpPclassbase\completeness\xspace}
% W[i]-hard
\newcommand{\Wcompleteness}[1]{\W{#1}-\completeness\xspace}

%%% ----------------------------------------------------------------------
%%% reduction
%%% ----------------------------------------------------------------------

\newcommand{\reduction}{reduction}
\newcommand{\reductions}{reductions}
\newcommand{\reductible}{reductible}

\newcommand{\APTypeReduction}{AP}
\newcommand{\PTASTypeReduction}{PTAS}
\newcommand{\LTypeReduction}{L}
\newcommand{\ETypeReduction}{E}
\newcommand{\fptTypeReduction}{fpt}
\newcommand{\pptTypeReduction}{ptp}

% AP-reduction
\newcommand{\APreduction}{\APTypeReduction-\reduction\xspace}
\newcommand{\APreductions}{\APTypeReduction-\reductions\xspace}
\newcommand{\APreductible}{\APTypeReduction-\reductible\xspace}

% PTAS-reduction
\newcommand{\PTASeduction}{\PTASTypeReduction-\reduction\xspace}
\newcommand{\PTASreductions}{\PTASTypeReduction-\reductions\xspace}
\newcommand{\PTASreductible}{\PTASTypeReduction-\reductible\xspace}

% L-reduction
\newcommand{\Lreduction}{\LTypeReduction-\reduction\xspace}
\newcommand{\Lreductions}{\LTypeReduction-\reductions\xspace}
\newcommand{\Lreductible}{\LTypeReduction-\reductible\xspace}

% E-reduction
\newcommand{\Ereduction}{\ETypeReduction-\reduction\xspace}
\newcommand{\Ereductions}{\ETypeReduction-\reductions\xspace}
\newcommand{\Ereductible}{\ETypeReduction-\reductible\xspace}

% fpt-reduction
\newcommand{\fptreduction}{\fptTypeReduction-\reduction\xspace}
\newcommand{\fptreductions}{\fptTypeReduction-\reductions\xspace}
\newcommand{\fptreductible}{\fptTypeReduction-\reductible\xspace}

% ptp-reduction
\newcommand{\ptpreduction}{\ptpTypeReduction-\reduction\xspace}
\newcommand{\ptpreductions}{\ptpTypeReduction-\reductions\xspace}
\newcommand{\ptpreductible}{\ptpTypeReduction-\reductible\xspace}

% symbols
\DeclareMathOperator{\APreduce}{\text{$\leq_{\text{\APTypeReduction}}$}}
\DeclareMathOperator{\PTASreduce}{\text{$\leq_{\text{\PTASTypeReduction}}$}}
\DeclareMathOperator{\Lreduce}{\text{$\leq_{\text{\LTypeReduction}}$}}
\DeclareMathOperator{\Ereduce}{\text{$\leq_{\text{\ETypeReduction}}$}}
\DeclareMathOperator{\fptreduce}{\text{$\leq_{\text{\fptTypeReduction}}$}}
\DeclareMathOperator{\ptpreduce}{\text{$\leq_{\text{\fptTypeReduction}}$}}

%% 
%% Approximation
%%
\DeclareMathOperator{\poly}{poly}
\DeclareMathOperator{\POLY}{poly}
\DeclareMathOperator{\SIZE}{size}
\newcommand{\sol}{{\sf sol}\xspace}
\newcommand{\PB}[1]{\textsf{\scshape{#1}}}
\newcommand{\OPTname}{opt}
\newcommand{\OPT}{\text{$\mathsf{\bf \OPTname}$}}
\newcommand{\OPTpb}[1]{\text{$\mathsf{\OPTname}_{\PB{#1}}$}}
\newcommand{\ALGO}[1]{\textbf{\ttfamily\sf #1}}
\newcommand{\Approxname}{Approx}
\newcommand{\APPROX}[1]{\text{$\ALGO{\Approxname}_{\,\PB{#1}}$}}
\newcommand{\PCP}{{\sf\bf PCP}\xspace}

%%
%% Problem Definition
%%
\newcommand{\PbDef}[3]{%
\begin{center}
  \begin{tabular}{l}%
    \shadowbox{%
    \begin{minipage}[c]{.9\textwidth}
      \smallskip%
      \par\noindent%
      {#1}%
      \smallskip
      \par\noindent%
      $\bullet$
      \textbf{\textsf{Input}}~: #2% 
      \medskip
      \par\noindent%
      $\bullet$
      \textbf{\textsf{Question}}~:
      #3% 
      \smallskip%
      \par\noindent%
    \end{minipage}
  }% end shadowbox
  \end{tabular}%
\end{center}
}%
\newcommand{\PbDefinition}{\PbDef}

%%
%% Problem (Input + Output) Definition
%%
\newcommand{\PbInputOutputDef}[3]{%
\begin{center}
  \begin{tabular}{l}%
    \shadowbox{%
    \begin{minipage}[c]{.9\textwidth}
      \smallskip%
      \par\noindent%
      \PB{#1}%
      \medskip%
      \par\noindent%
      $\bullet$
      \textbf{\textsf{Input}}~: #2% 
      \medskip
      \par\noindent%
      $\bullet$
      \textbf{\textsf{Output}}~:
      #3% 
      \smallskip%
      \par\noindent%
    \end{minipage}
  }% end shadowbox
  \end{tabular}%
\end{center}
}%
\newcommand{\PbInputOutputDefinition}{\PbInputOutputDef}


%%
%% Optimization Problem Definition
%%

\newcommand{\OptPbDefinition}[4]{%
\begin{center}
  \begin{tabular}{l}%
    \shadowbox{%
    \begin{minipage}[c]{.9\textwidth}
      \par\noindent%
      \shadowbox{#1}%
      \par\noindent%
      $\bullet$
      \textbf{\textsf{Input}}~: #2% 
      \par\noindent%
      $\bullet$
      \textbf{\textsf{Solution}}~: #3%  
      \par\noindent%
      $\bullet$
      \textbf{\textsf{Measure}}~: #4% 
      \par\noindent%
    \end{minipage}
    }% end shadowbox
  \end{tabular}%
\end{center}
}%

%%
%% Parameterized Problem Definition
%%
\newcommand{\ParamPbDefinition}[4]{%
\begin{center}
  \begin{tabular}{l}%
    %\shadowbox{%
    \begin{minipage}[c]{.95\textwidth}
      % \smallskip%
      \par\noindent%
      #1%
      % \smallskip%
      \par\noindent%
      \textbf{\textsf{Input}}~: #2% 
      % \smallskip
      \par\noindent%
      \textbf{\textsf{Question}}~: #3%  
      % \smallskip
      \par\noindent%
      \textbf{\textsf{Parameter}}~: #4% 
      %\smallskip%
      \par\noindent%
    \end{minipage}
  %}% end shadowbox
  \end{tabular}%
\end{center}
}%
\newcommand{\ParamPbDefinitionTwo}[5]{%
\begin{center}
  \begin{tabular}{l}%
    \shadowbox{%
    \begin{minipage}[c]{.9\textwidth}
      \smallskip%
      \par\noindent%
      \shadowbox{#1}%
      \medskip%
      \par\noindent%
      $\bullet$
      \textbf{\textsf{Input}}~: #2% 
%      \medskip
      \par\noindent%
      $\bullet$
      \textbf{\textsf{Parameter}}~: #3% 
 %     \medskip
      \par\noindent%
      $\bullet$
      \textbf{\textsf{Parameter}}~: #4% 
      \medskip
      \par\noindent%
      $\bullet$
      \textbf{\textsf{Question}}~: #5%  
      \smallskip%
      \par\noindent%
    \end{minipage}
    }% end shadowbox
  \end{tabular}%
\end{center}
}%
\newcommand{\ParamPbDefinitionThree}[6]{%
\begin{center}
  \begin{tabular}{l}%
    \shadowbox{%
    \begin{minipage}[c]{.9\textwidth}
      \smallskip%
      \par\noindent%
      \shadowbox{#1}%
      \medskip%
      \par\noindent%
      $\bullet$
      \textbf{\textsf{Input}}~: #2% 
%      \medskip
      \par\noindent%
      $\bullet$
      \textbf{\textsf{Parameter}}~: #3% 
%      \medskip
      \par\noindent%
      $\bullet$
      \textbf{\textsf{Parameter}}~: #4% 
%      \medskip
      \par\noindent%
      $\bullet$
      \textbf{\textsf{Parameter}}~: #5%  
      \medskip
      \par\noindent%
      $\bullet$
      \textbf{\textsf{Question}}~: #6%  
      \smallskip%
      \par\noindent%
    \end{minipage}
    }% end shadowbox
  \end{tabular}%
\end{center}
}%
\newcommand{\ParamPbDefinitionFour}[7]{%
\begin{center}
  \begin{tabular}{l}%
    \shadowbox{%
    \begin{minipage}[c]{.9\textwidth}
      \smallskip%
      \par\noindent%
      \shadowbox{#1}%
      \medskip%
      \par\noindent%
      $\bullet$
      \textbf{\textsf{Input}}~: #2% 
%      \medskip
      \par\noindent%
      $\bullet$
      \textbf{\textsf{Parameter}}~: #3% 
%      \medskip
      \par\noindent%
      $\bullet$
      \textbf{\textsf{Parameter}}~: #4% 
%      \medskip
      \par\noindent%
      $\bullet$
      \textbf{\textsf{Parameter}}~: #5%  
%      \medskip
      \par\noindent%
    $\bullet$
      \textbf{\textsf{Parameter}}~: #6%  
      \medskip
      \par\noindent%
      $\bullet$
      \textbf{\textsf{Question}}~: #7%  
      \smallskip%
      \par\noindent%
    \end{minipage}
    }% end shadowbox
  \end{tabular}%
\end{center}
}%
\newcommand{\ParamPbDefinitionFive}[8]{%
\begin{center}
  \begin{tabular}{l}%
    \shadowbox{%
    \begin{minipage}[c]{.9\textwidth}
      \smallskip%
      \par\noindent%
      \shadowbox{#1}%
      \medskip%
      \par\noindent%
      $\bullet$
      \textbf{\textsf{Input}}~: #2% 
%      \medskip
      \par\noindent%
      $\bullet$
      \textbf{\textsf{Parameter}}~: #3% 
%      \medskip
      \par\noindent%
      $\bullet$
      \textbf{\textsf{Parameter}}~: #4% 
%      \medskip
      \par\noindent%
      $\bullet$
      \textbf{\textsf{Parameter}}~: #5%  
%      \medskip
      \par\noindent%
    $\bullet$
      \textbf{\textsf{Parameter}}~: #6%  
%      \medskip
      \par\noindent%
    $\bullet$
      \textbf{\textsf{Parameter}}~: #7%  
      \medskip
      \par\noindent%
      $\bullet$
      \textbf{\textsf{Question}}~: #8%  
      \smallskip%
      \par\noindent%
    \end{minipage}
    }% end shadowbox
  \end{tabular}%
\end{center}
}%
\newcommand{\ParamPbDefinitionSix}[9]{%
\begin{center}
  \begin{tabular}{l}%
    \shadowbox{%
    \begin{minipage}[c]{.9\textwidth}
      \smallskip%
      \par\noindent%
      \shadowbox{#1}%
      \medskip%
      \par\noindent%
      $\bullet$
      \textbf{\textsf{Input}}~: #2% 
%      \medskip
      \par\noindent%
      $\bullet$
      \textbf{\textsf{Parameter}}~: #3% 
%      \medskip
      \par\noindent%
      $\bullet$
      \textbf{\textsf{Parameter}}~: #4% 
%      \medskip
      \par\noindent%
      $\bullet$
      \textbf{\textsf{Parameter}}~: #5%  
%      \medskip
      \par\noindent%
      $\bullet$
      \textbf{\textsf{Parameter}}~: #6%  
%      \medskip
      \par\noindent%
      $\bullet$
      \textbf{\textsf{Parameter}}~: #7%  
%      \medskip
      \par\noindent%
      $\bullet$
      \textbf{\textsf{Parameter}}~: #8%  
      \medskip
      \par\noindent%
      $\bullet$
      \textbf{\textsf{Question}}~: #9%  
      \smallskip%
      \par\noindent%
    \end{minipage}
    }% end shadowbox
  \end{tabular}%
\end{center}
}%




%%%%%%%%%%%%%%%%%%%%%%%%%%%%%%%%%%%%%%%%%%%%%%%%%%%%%%%%%%%%%%%%%%%%%%%%%%%%%%%

% \renewcommand{\floatpagefraction}{.9}
% \renewcommand{\textfraction}{.1}

%%%%%%%%%%%%%%%%%%%%%%%%%%%%%%%%%%%%%%%%%%%%%%%%%%%%%%%%%%%%%%%%%%%%%%%%%%%%%%%


\newcommand{\PET}{pet}
\newcommand{\SBL}{sbl}
\newcommand{\horiz}{{\centering\rule{350pt}{.4pt}}}

%%%%%%%%%%%%%%%%%%%%%%%%%%%%%%%%%%%%%%%%%%%%%%%%%%%%%%%%%%%%%%%%%%%%%%%%%%%%%%%

\makeatletter
\newcommand{\pushright}[1]{\ifmeasuring@#1\else\omit\hfill$\displaystyle#1$\fi\ignorespaces}
\newcommand{\pushleft}[1]{\ifmeasuring@#1\else\omit$\displaystyle#1$\hfill\fi\ignorespaces}
\makeatother

%%%%%%%%%%%%%%%%%%%%%%%%%%%%%%%%%%%%%%%%%%%%%%%%%%%%%%%%%%%%%%%%%%%%%%%%%%%%%%%

\begin{document}

%%%%%%%%%%%%%%%%%%%%%%%%%%%%%%%%%%%%%%%%%%%%%%%%%%%%%%%%%%%%%%%%%%%%%%%%%%%%%%%

\title{%
Obtaining a Triangular Matrix by Independent Row-Column Permutations
}%

\author{%
  Guillaume Fertin\inst{1} \and
  Irena Rusu\inst{1} \and
  St\'ephane Vialette\inst{2}
}% end author
\institute{%
  LINA UMR CNRS 6241, Universit\'e de Nantes, France \\
  \email{\{guillaume.fertin,irena.rusu\}@univ-nantes.fr}
  \and
  Universit\'e Paris-Est, LIGM (UMR 8049), CNRS, UPEM, ESIEE Paris, ENPC, 
  F-77454, Marne-la-Vallée, France\\
  \email{vialette@univ-mlv.fr}
}% end institute
\date{\today}

\maketitle

%%%%%%%%%%%%%%%%%%%%%%%%%%%%%%%%%%%%%%%%%%%%%%%%%%%%%%%%%%%%%%%%%%%%%%%%%%%%%%%

\begin{abstract}
Given a square $(0,1)$-matrix $A$,
we consider the problem of deciding whether there exists a permutation
of the rows and a permutation of the columns of 
$A$ such that after carrying out these permutations,
the resulting matrix is triangular. The complexity of the problem was
posed as an open question by Wilf~\cite{Wilf:1997} in 1997.
%(\emph{i.e.}, do there exist permutation matrices $P$ and $Q$ such that
%$PAQ$ is triangular)?
In 1998, DasGupta et al.~\cite{DasGuptaJiangKannanLiSweedyk:1998} seemingly answered the
question, proving it is \NPC. However, we show here that their result is flawed, which 
leaves the question still open. Therefore, we give a definite answer
to this question by proving that the problem is \NPC. We finally present an
exponential-time algorithm for solving the problem.
\end{abstract}

%%%%%%%%%%%%%%%%%%%%%%%%%%%%%%%%%%%%%%%%%%%%%%%%%%%%%%%%%%%%%%%%%%%%%%%%%%%%%%%



%%
%%
%% ---- Introduction ----
%%
\section{Introduction}
\label{section:Introduction}

In his contribution in the tribute to the late Professor
Erd\"os~\cite{Wilf:1997}, Wilf posed the following question:
\emph{
  ``Let $A$ be a $m \times n$ matrix of $0$'s and $1$'s. Consider the
  computational problem: do there exist permutations $P$ of the rows
  of $A$, and $Q$, of the columns of $A$ such that after carrying out
  these permutations, $A$ is triangular? The question we ask concerns
  the complexity of the problem.
  Is this problem \NPC?
  Or, does there exist a polynomial-time algorithm for doing it?''}
As noted by Wilf,
this problem is strongly related to job scheduling with precedence
constraints, a well-known problem in theoretical computer science.
The present paper is devoted to giving an answer to this question.

A square matrix is called {\em lower triangular} if all the entries above the main diagonal are zero.
Similarly, a square matrix is called {\em upper triangular} if all the entries below the main diagonal
are zero.
A {\em triangular matrix} is one that is either lower triangular or upper triangular.
Because matrix equations with triangular matrices are easier to solve,
they are very important in  linear algebra and numerical analysis.
We refer the reader to \cite{MatrixComputations:1996} for a further discussion.

For an arbitrary square matrix $A$, it is well-known in linear algebra
that there exists an invertible matrix $S$ such that $S^{-1}AS$ is
upper triangular.
We focus here, however, on permutation matrices.
Recall that a permutation matrix is a square matrix obtained from the same
size identity matrix by a permutation of rows. 
A product of permutation matrices is again a permutation matrix and
the inverse of a permutation matrix is again a permutation matrix.
In fact, for any permutation matrix $P$, $P^{-1} = P^T$.

% Nicolson [5] shows that for a nonnegative matrix $A$ the following four conditions
% are equivalent:
% (i) there exists a permutation matrix $P$ such that
% $P A P^T$ is strictly upper triangular;
% (ii) there is no positive cycle in $M$
% (i.e. in the weighted digraph there is no cycle whose edges all have positive weight);
% (iii) the permanent of $M + I$ is $1$; and
% (iv) $A$ is nilpotent.


This paper is organized as follows.
In Section~\ref{section:Notations}, we provide the basic material
needed for this paper. %with a
%strong emphasis on the associated combinatorial structures: configurations of subsets,
%bipartite graphs and directed graphs.
Section~\ref{section:Hardness result} is devoted to proving hardness of
determining whether a square $(0,1)$-matrix is permutation equivalent
triangular, i.e. whether it can be transformed into a triangular matrix by independent
row and column permutations.
In Section~\ref{section:Exponential-time algorithm}, we give some properties of
permutation equivalent triangular matrices (or \PET\ matrices, for short) and present 
an exponential-time algorithm to determine whether a matrix is a \PET\ matrix.
The paper concludes with suggestions for further research directions.


%%%%%%%%%%%%%%%%%%%%%%%%%%%%%%%%%%%%%%%%%%%%%%%%%%%%%%%%%%%%%%%%%%%%%%%%%%%%%%%

%%
%% ---- Preliminaries ----
%%
\section{Notations}
\label{section:Notations}

%\subsection{$(0,1)$-matrices}
%\subsubsection{$(0,1)$-matrices} 
For any positive integer $n$, denote $[n]=\{1, 2, \ldots, n\}$. 
Let $A = [a_{i,j}]$, $1 \leq i \leq m$ and $1 \leq j \leq n$,
be a matrix of $m$ rows and $n$ columns.
% We say that $A$ is of size $m$ by $n$ and we also refer to $A$
% as an $m$ by $n$ matrix.
In the case that $m = n$ then the matrix is \emph{square} of \emph{order} $n$.
It is always assumed that the entries of the matrix are elements
of some underlying field $F$.
It is convenient to refer to either a row or a column of the matrix as a
\emph{line} of the matrix.
We use the notation $A^T$ for the \emph{transpose} of matrix $A$.
We always designate a zero matrix by $\zeros$, a matrix with every entry equal to
$1$ by $J$, and the identity matrix of order $n$ by $I$.
In order to emphasize the size of these matrices we sometimes include subscripts.
Thus $J_{m,n}$ denotes the all $1$'s matrix of size $m$ by $n$,
and this is abbreviated to $J_n$ if $m = n$. 
Notations $\zeros_{m,n}$, $\zeros_n$ and $I_n$ are similarly defined.
In displaying a matrix we often use $*$ to designate a submatrix of no particular structure.
Two matrices $A$ and $B$ are said to be \emph{permutation equivalent},
denoted by $A \sim B$, if there exist permutation matrices $P$ and $Q$ of
suitable sizes such that $B = PAQ$.

We will be greatly concerned with matrices whose entries consist exclusively of
the integers $0$ and $1$.
Such matrices are referred to as \emph{$(0,1)$-matrices}.
For a $(0,1)$-matrix $A$, we let $\omega(A)$ stand for the number of $1$'s in $A$.
A square matrix $A = [a_{i,j}]$ of order $n$ is said to be
\emph{lower left triangular} if it has only $0$'s above the main diagonal
(\emph{i.e.} $a_{i,j} = 0$ for $1 \leq i < j \leq n$.
We write $\lltm_{\,n}$ for the lower left triangular $(0,1)$-matrix 
whose $0$'s are exclusively above the main diagonal.
For two matrices $A = [a_{i,j}]$ and $B = [b_{i,j}]$ of size $m$ by $n$,
we write $A \leq B$ if $a_{i,j} \leq b_{i,j}$ for $1 \leq i \leq m$
and $1 \leq j \leq n$, so that a square matrix $A$ of order $n$ is
lower left triangular if $A \leq \lltm_{\,n}$.
In the context of permutation equivalent matrices,
we will sometimes not be interested in any particular orientation of a
triangular matrix and forget about any
specific orientation such as ``\emph{lower left}''.
Furthermore, for readability, a matrix which is permutation equivalent
to a triangular matrix is said to be a \emph{\PET\ matrix}.
The \emph{row sum vector}
$\RSV(A) = \begin{bmatrix} r_1 & r_2 & \ldots & r_m \end{bmatrix}$
and the  \emph{column sum vector}
$\CSV(A) = \begin{bmatrix} c_1 & c_2 & \ldots & c_n \end{bmatrix}$
of $A$ are defined by
$r_i = \sum_{1 \leq j \leq n} a_{i,j}$ for $1 \leq i \leq m$
and
$c_j = \sum_{1 \leq i \leq m} a_{i,j}$ for $1 \leq j \leq n$.
The row sum vector $\RSV(A)$
(resp. column sum vector $\CSV(A)$)
is \emph{stepwise bounded}
if  $|\{i : r_i \leq k \}| \geq k$
(resp. $|\{j : c_j \leq k \}| \geq k$)
for $1 \leq k \leq n$.
It is clear that if a $(0,1)$-matrix $A$ is a \PET\ matrix then
both $\RSV(A)$ and $\CSV(A)$ are stepwise bounded.

% The definition of the \emph{permanent} of $A$ follows the classical definition of the
% determinant but with a simplification. Ironically, this simplification in the formula makes
% it more difficult to compute the permanent.
The permanent of $A = [a_{i,j}]$ is defined as the number given by the formula
$\PER(A) = \sum_{(j_1, j_2, \ldots, j_n) \in S_n}
a_{1,j_1}\,a_{2,j_2}\, \ldots a_{n,j_n}$,
where the summation is over all permutations
$(j_1,j_2, \ldots, j_n)$ of $[n]$.
Observe that, unlike the determinant, we do not put a minus sign in front of some of the terms in the
summation.
Of particular importance,
the permanent does not change if we permute the rows of $A$ and
permute the columns of $A$.

Let $A = [a_{i,j}]$ be an $m$ by $n$ matrix.
For convenience, for a set $K \subseteq [m]$ we will
write $\overline{K}$ for the set
$[m] \setminus K$.
Let $K = \{i_1, i_2, \ldots, i_k\}$ be a set of $k$ elements with
$K \subseteq [m]$,
and let $L = \{j_1, j_2, \ldots, j_l\}$ be a set of
$l$ elements with $L \subseteq [n]$.
The sets $K$ and $L$ designate a collection of row indices and column indices,
respectively, of the matrix $A$, and the $k$ by $l$ submatrix determined by them is
denoted $A[K,L]$.
% $$
% A[K,L]=
% \begin{bmatrix}
% a_{i_1,j_1} & a_{i_1,j_2} & \dots  & a_{i_1,j_l} \\
% a_{i_2,j_1} & a_{i_2,j_2} & \dots  & a_{i_2,j_l} \\
% \vdots      & \vdots      & \ddots & \vdots      \\
% a_{i_k,j_1} & a_{i_k,j_2} & \dots  & a_{i_k,j_l}
% \end{bmatrix}\text{.}
% $$
% If $L = K$, then $A[K, K]$ is a \emph{principal} submatrix of $A$,
% sometimes denoted more simply as $A[K]$.


%\subsection{Related combinatorial objects}
%\subsubsection{Configuration of subsets} 

% Combinatorial matrix theory \cite{Brualdi:Ryser:1991} includes a wide variety of 
% combinatorial areas, e.g. graph theory \cite{Diestel:2000}, transversal theory
% \cite{Mirsky:1971}, combinatorial configurations \cite{Ryser:1969}.
% We shall make intensive use of configuration of subsets.

Let $X = \{x_i : 1 \leq i \leq n\}$ be a non-empty set of $n$
elements, that we call an \emph{$n$-set}.
Let $\mathcal{S} = (S_i : 1 \leq i \leq m)$ be $m$ not necessarily
distinct
subsets of the $n$-set $X$. We refer to this collection of subsets of an
$n$-set as a \emph{configuration of subsets}.
We set $a_{i,j} = 1$ if $x_j \in S_i$, and $a_{i,j} = 0$
if $x_i \notin S_i$.
The resulting $(0,1)$-matrix $A = [a_{i,j}]$, $1 \leq i \leq m$ and
$1 \leq j \leq n$ of size $m$ by $n$ is the \emph{incidence matrix}
for the configuration of subsets $\mathcal{S}$ of the $n$-set $X$.
The $1$s in row $\alpha_i$ of $A$ display the elements in the subset
$S_i$, and the $1$'s in column $\beta_j$ display the occurrences of $x_j$
among the subsets. 
Let $\mathcal{S} = (S_i : 1 \leq i \leq n)$ be a configuration of subsets of some 
ground $n$-set $X$. A bijective mapping $\varphi : \mathcal{S} \to [n]$ is said to be a
\emph{stepwise bounded labeling} (or \SBL\ for short) of $\mathcal{S}$ if
$\left|\bigcup_{\varphi(S_j) \leq i}S_j\right| \leq i$
for $1 \leq i \leq n$.

%  A \emph{system of distinct representatives} (SDR) for the family $\mathcal{S}$
% is a family $(x_1, x_2, \ldots, x_n)$ of elements of $X$ satisfying
% the conditions
% (i) $x_i \in S_i$ for $1 \leq i \leq n$, and
% (ii) $x_i \neq xj$ for $i \neq j$.
% (The first condition asserts that the elements are representatives of the sets,
% and the second that they are distinct.)
% The collection $\mathcal{S}$ satisfies the \emph{marriage} condition
% if and only if for each subcollection $\mathcal{S}' \subseteq \mathcal{S}$, we have
% $|\mathcal{S}'| \leq \left|\bigcup_{S \in \mathcal{S}'} S\right|$.
% In other words, the number of sets in each subcollection $\mathcal{S}'$ is less
% than or equal to the number of distinct elements in the union over the
% subcollection $\mathcal{S}'$.
% Hall's theorem states that $\mathcal{S}$ has a SDR if and only if
% $\mathcal{S}$ satisfies the marriage condition.

% \paragraph{Bipartite graphs.}

% A graph $G=(V,E)$ is \emph{bipartite} if 
% $V$ can be partitioned into two sets $U$ and $W$ such that each edge
% joins a vertex in $U$ to a vertex in $W$, and the pair $(U, W)$ is called a
% \emph{bipartition} of $V$ (or of $G$).
% The \emph{reduced adjacency matrix} $B = [b_{i,j}]$ is an $m \times n$
% $(0,1)$-matrix defined by $b_{i, j} = 1$ if and only if
% there is an edge beteween vertex $i$ of $U$ and vertex $j$ of $W$.
% A matching $M$
% in $G$ is a collection of edges no two of which have a vertex in
% common. If $v$ is a vertex and there is an edge of $M$ containing $v$,
% then $v$ \emph{meets} the matching $M$ and the matching $M$ meets the vertex
% $v$. A \emph{perfect matching} of $G$ is a matching
% that meets all vertices of $G$.

% \paragraph{Directed graphs.}

% A \emph{directed graph} (for short, a \emph{digraph}) $G$
% consists of a finite set $V$ of elements called vertices and a set $E$ of
% ordered pairs of vertices called arcs.
% The \emph{adjacency matrix} associated to $G$ is the $|V| \times |V|$
% $(0,1)$-matrix $A = [a_{i,j}]$ defined by $a_{i,j} = 1$ if and only if
% there is an arc from vertex $i$ to vertex $j$ in $G$.


%%%%%%%%%%%%%%%%%%%%%%%%%%%%%%%%%%%%%%%%%%%%%%%%%%%%%%%%%%%%%%%%%%%%

%%
%% ---- Properties ----
%%
\section{Answering Wilf's question}%Hardness result}
\label{section:Hardness result}

We prove in this section that, given a square $(0,1)$-matrix $A$, deciding whether there 
exists a permutation matrix $P$ and a permutation matrix $Q$ of suitable size
such that $PAQ$ is triangular is \NPC.

\subsection{Disproving a previous related result}

Before giving our proof, it is worth mentioning that the following problem
(called LBQIS$(n,k)$ and rephrased to fit the context of this paper) is
claimed to be \NPC in~\cite{DasGuptaJiangKannanLiSweedyk:1998}:
Given a $(0,1)$-matrix of order $n$ and positive integer $k \leq n$,
do there exist permutation matrices $P$ and $Q$ such that
$
PAQ =
\begin{bmatrix}
A_{1,1} & A_{1,2} \\
A_{2,1} & A_{2,2}
\end{bmatrix}
$
with $A_{1,2}$ a square lower triangular matrix of size $k$ by $k$?
It is not very difficult to find a polynomial transformation from
LBQIS to Wilf's question, which would prove the \NPcompleteness of the
latter. Just add $n-k$ empty rows and $n-k$ empty columns 
to matrix $A$ to obtain a new matrix $A'$. Now, notice that each 
submatrix $A_{1,2}$ in a solution for LBQIS may be completed with the 
$n-k$ empty rows put before row 1 of $A_{1,2}$ and with the $n-k$ empty 
columns put after column $k$ of $A_{1,2}$ to yield a solution for the 
instance $A'$ in Wilf's question, and viceversa. 

Unfortunately, paper~\cite{DasGuptaJiangKannanLiSweedyk:1998} 
contains a serious flaw in the proof. To fix things, note that
in~\cite{DasGuptaJiangKannanLiSweedyk:1998} LBQIS is stated in
terms of bipartite graphs, for which matrix $A$ is the reduced adjacency
matrix. Then, LBQIS$(n,k)$ is proved \NPC by reduction 
from another problem on bipartite graphs called LBIS$(n,k)$, using the so-called 
Rearrangement Lemma (Lemma 3.5 in ~\cite{DasGuptaJiangKannanLiSweedyk:1998}).
Two affirmations in the proof of this lemma are contradicted by the following
example. Let $G$ be the graph (input for LBIS) with vertices $U=\{i\, |\, 1\leq i\leq 4\}$
and $V=\{i\, |\, 1\leq i\leq 4\}$, whose edges are $(1,1), (2,1), (2,2),
(3,2), (3,4), (4,3)$ and $(4,4)$. Thus, $n=4$. Define $k=1$.  
Let $G'$ be the input graph for LBQIS built as in~\cite{DasGuptaJiangKannanLiSweedyk:1998},
and $k'=k^2+k=2$. Then the vertex subset $U'\cup V'$ of $G'$, with $U'=\{[2,4], [1,2]\}$
and $V'=\{[1,3], [2,2]\}$ is a solution of LBQIS of size $k'$ for which the
assumption on the first line of the Rearrangement Lemma's proof is false.
Also, the vertex subset $U'\cup V'$ of $G'$, with $U'=\{[1,1], [2,1]\}$
and $V'=\{[1,2], [1,3]\}$ is a solution of LBQIS of size $k'$ for which
the second affirmation in the same lemma (``clearly $q_1\leq p_1$'') is also false.

\subsection{Our \NPcompleteness proof for Wilf's question}

We present our results in terms of \SBL\ for configurations
of subsets. The rationale for considering \SBL\ for configurations
of subsets stems from the following lemma.

\begin{lemma}
    \label{lemma:bounded stepwise property <=> PAQ triangular}
    Let $\mathcal{S} = (S_i : 1 \leq i \leq n)$ be a configuration of subsets of
    some ground $n$-set, and let $A$ be the corresponding incidence matrix.
    There exist permutation matrices $P$ and $Q$ of order $n$
    such that $PAQ \leq \lltm_{\,n}$ if and only if
    there exists an \SBL\ of $\mathcal{S}$.
\end{lemma}

We need to focus our attention on a special type of \SBL. Call a bijective mapping
$\varphi : \mathcal{S} \to [n]$ \emph{normalized} if $\varphi$
maps the identical subsets of elements of $\mathcal{S}$
to a set of consecutive integers.
Most of the interest in normalized bijective labelings
stems from the following intuitive lemma.

\begin{lemma}
  \label{lemma:normalized stepwise bounded}
  Let $\mathcal{S} = (S_i : 1 \leq i \leq n)$ be a configuration of subsets of
    some ground $n$-set.
    If there exists an \SBL\ of $\mathcal{S}$
    then there exists a normalized \SBL\ of $\mathcal{S}$.
\end{lemma}

We are now ready to prove that
deciding whether there exists an \SBL\ of some
configuration of subsets is \NPC thereby proving that 
deciding whether a square $(0,1)$-matrix is a \PET\ matrix is \NPcomplete as well.
The proof proceeds by a reduction from the \PB{3Sat} problem - which is a known
\NPcomplete problem \cite{Cook:1971}.
% (see also
% \cite{Garey:Johnson:1979} and \cite{Papadimitriou:1994} for variants).
Let an arbitrary
instance of the \PB{3Sat} problem be given by a $3$CNF formula 
$\phi = c_1 \vee c_2 \vee \ldots \vee c_m$ over variables 
$x_1, x_2, \ldots, x_n$. 
Our construction is divided into two steps: 
(1) construction of a (polynomial size) ground set $\mathbf{X}$ and 
(2) construction of a configuration of subsets $C$ of the ground set
$\mathbf{X}$. 
Throughout the proof, parts of the ground set
$\mathbf{X}$ are written as capital bold letters 
($\mathbf{V}, \mathbf{T}, \mathbf{F}, \ldots$)
and subsets of the configuration are written with capital calligraphic
letters ($\mathcal{V}_i, \mathcal{T}_i, \mathcal{F}_i, \ldots$). 

  \begin{figure}[H]
    \centering
    \begin{minipage}[c]{\textwidth}
      \centering
      \input{r.pstex_t}
      \caption{\label{fig:stepwise bounded labeling construction}
          Illustration of the construction 
          % used in the proof of 
          % Proposition~\ref{proposition:Stepwise Bounded Labeling is NPC} 
          for the $3$CNF formula 
          $\phi = (x_1 \vee x_2 \vee x_3) \wedge 
          (\overline{x}_1 \vee \overline{x}_2 \vee x_3) \wedge 
          (\overline{x}_1 \vee x_2 \vee \overline{x}_3)$. 
          Identical subsets are not distinguishable
          in our representation. A satisfying truth assignment is
          given by $f(x_1) = \mbox{TRUE}$, $f(x_2) = \mbox{FALSE}$ and
          $f(x_3) = \mbox{FALSE}$. For sake of clarity, neither
          the ground set $\mathbf{X}$ nor the collection of subsets
          $C$ is fully represented. 
          % Note that we have here dropped the
          % assumption that each clause is satisfied by its first
          % literal
          } 
    \end{minipage}%
  \end{figure}

To begin with, define 
$p_i = 3(n + m + 1 - i) + 2$,
$q_i = 3(n + m + 1 - i) + 1$ and
$r_i = 3(n + m + 1 - i)$ for $1 \leq i \leq n + m$.
Furthermore, define $p_{n+m+1}=1$,
$K = \sum_{i=1}^{n} q_i + 2 \sum_{i=n+1}^{n+m} q_i$ and
$L = \sum_{i=1}^{n+m} (p_{i+1} + r_i)$.
Let us now define the ground set $\mathbf{X}$. 
Consider the pairwise disjoint sets defined as follows:
$\mathbf{V}_i = \{v_{i, j} \mid 1 \leq j \leq p_i\}$,
$\mathbf{V}'_i  = \{v'_{i, j} \mid 1 \leq j \leq r_i\}$,
$\mathbf{T}_i = \{t_{i, j} \mid 1 \leq j \leq q_i\}$,
$\mathbf{F}_i = \{f_{i, j} \mid 1 \leq j \leq q_i\}$
for $1 \leq i \leq n$. Furthermore, define
$\mathbf{C}_i = \{c_{i, j} \mid 1 \leq j \leq p_{n + i}\}$
$\mathbf{C}'_i = \{c'_{i, j} \mid 1 \leq j \leq r_{n + i}\}$
for $1 \leq i \leq m$, and
$\mathbf{L}_{i, k} = \{\ell_{i, k, j} \mid 1 \leq j \leq q_{n + i}\}$
for $1 \leq i \leq m$ and $1 \leq k \leq 3$.
Finally, define $\mathbf{S} = \{s\}$.
For simplicity of notation, write
$\mathbf{V} = \bigcup_{1 \leq i \leq n} \mathbf{V}_i$,
$\mathbf{V}'= \bigcup_{1 \leq i \leq n} \mathbf{V}'_i$,
$\mathbf{T} = \bigcup_{1 \leq i \leq n} \mathbf{T}_i$,
$\mathbf{F} = \bigcup_{1 \leq i \leq n} \mathbf{F}_i$,
$\mathbf{C} = \bigcup_{1 \leq i \leq m} \mathbf{C}_i$,
$\mathbf{C}' = \bigcup_{1 \leq i \leq m} \mathbf{C}'_i$,
and
$\mathbf{L}_i = \bigcup_{1 \leq k \leq 3} \mathbf{L}_{i, k}$
for $1 \leq i \leq m$
and
$\mathbf{L} = \bigcup_{1 \leq i \leq m} \mathbf{L}_i$.
Informally, elements of $\mathbf{V} \cup \mathbf{V}'$ are associated to variables, 
elements of $\mathbf{T} \cup \mathbf{F}$ are associated to literals, elements of 
$\mathbf{C} \cup \mathbf{C}'$ are associated to clauses, elements of $\mathbf{L}$ 
are associated to literals in clauses and $\mathbf{S}$ is a  
\emph{separator set}. The ground set $\mathbf{X}$ of our construction is defined to be
$\mathbf{X} = 
\mathbf{V} \cup \mathbf{V}' \cup \mathbf{T} \cup 
\mathbf{F} \cup \mathbf{C} \cup \mathbf{C}' \cup 
\mathbf{L} \cup \mathbf{S}$.

Having defined the ground set $\mathbf{X}$, 
we now turn to the detailed construction of a configuration of subsets $C$ of $\mathbf{X}$. 
For sake of clarity, this will be divided into several steps. 
First, each variable $x_i$, $1 \leq i \leq n$, is associated to identical subsets 
$\mathcal{V}_{i, j}$, $1 \leq j \leq q_i$, in $C$. These subsets are defined as follows:
$\mathcal{V}_{i, j} = 
\left( \bigcup_{1 \leq k \leq i} \mathbf{V}_k \right) \cup 
\left( \bigcup_{1 \leq k \leq i-1} \mathbf{V}'_k \right)$
for $1 \leq i \leq n$ and $1 \leq j \leq q_i$.
Let us denote by $\mathcal{V}_i$, $1 \leq i \leq n$, 
the collection $(\mathcal{V}_{i, j} \mid 1 \leq j \leq q_i)$.
Next, each (positive) literal $x_i$, $1 \leq i \leq n$, 
is associated to identical subsets $\mathcal{T}_{i, j}$, $1 \leq j \leq r_i$, 
and to identical subset $\mathcal{T}'_{i, j}$, $1 \leq j \leq p_{i + 1}$. 
These subsets are defined as follows:
$\mathcal{T}_{i, j} = \mathbf{T}_i \cup 
\left( \bigcup_{1 \leq k \leq i} \mathbf{V}_k \right) \cup 
\left( \bigcup_{1 \leq k \leq i-1} \mathbf{V}'_k \right)$
for $1 \leq i \leq n$ and $1 \leq j \leq r_i$, and
$\mathcal{T}'_{i, j} = \mathbf{T}_i \cup \left( 
\bigcup_{1 \leq k \leq i} \mathbf{V}_k \right) \cup 
\left( \bigcup_{1 \leq k \leq i} \mathbf{V}'_k \right)$
for $1 \leq i \leq n$ and $1 \leq j \leq p_{i+1}$.
Of course, a similar construction of subsets applies for the negation 
$\overline{x}_i$ of each variable $x_i$, \emph{i.e.},
$\mathcal{F}_{i, j} = 
\mathbf{F}_i \cup \left( \bigcup_{1 \leq k \leq i} 
\mathbf{V}_k \right) \cup \left( \bigcup_{1 \leq k \leq i-1} 
\mathbf{V}'_k \right)$ for $1 \leq i \leq n$ and $1 \leq j \leq r_i$, and
$\mathcal{F}'_{i, j} = \mathbf{F}_i \cup \left( 
\bigcup_{1 \leq k \leq i} \mathbf{V}_k \right) \cup 
\left( \bigcup_{1 \leq k \leq i} \mathbf{V}'_k \right)$ 
for $1 \leq i \leq n$ and $1 \leq j \leq p_{i+1}$.
For readability, write 
$\mathcal{T}_i = (\mathcal{T}_{i, j} \mid 1 \leq j \leq r_i)$, 
$\mathcal{T}'_i = (\mathcal{T}'_{i, j} \mid 1 \leq j \leq p_{i+1})$, 
$\mathcal{F}_i = (\mathcal{F}_{i, j} \mid 1 \leq j \leq r_i)$ and 
$\mathcal{F}'_i = (\mathcal{F}'_{i, j} \mid 1 \leq j \leq p_{i+1})$ for $1 \leq i \leq n$. 
Note that the following (strict) inclusions hold for all $1 \leq i \leq n$, 
$1 \leq j_1 \leq q_i$, $1 \leq j_2 \leq r_i$ and $1 \leq j_3 \leq p_{i+1}$: 
(i) $\mathcal{V}_{i, j_1} \subset \mathcal{T}_{i, j_2} \subset \mathcal{T}'_{i, j_3}$ and 
(ii) $\mathcal{V}_{i, j_1} \subset \mathcal{F}_{i, j_2} \subset \mathcal{F}'_{i, j_3}$.
We now turn to the $m$ clauses of the $3$CNF formula. 
Each clause $c_i$, $1 \leq i \leq m$, is associated to identical subsets 
$\mathcal{C}_{i, j}$, $1 \leq j \leq q_{n+i}$. These subsets are defined as follows:
$\mathcal{C}_{i, j} = \mathbf{V} \cup \mathbf{V}' \cup 
\left( \bigcup_{1 \leq k \leq i} \mathbf{C}_k \right) 
\cup \left( \bigcup_{1 \leq k \leq i-1} \mathbf{C}'_k \right)$
for $1 \leq i \leq m$ and $1 \leq j \leq q_{n+i}$.
Let us denote by $\mathcal{C}_i$, $1 \leq i \leq m$, 
the collection $(\mathcal{C}_{i, j} \mid 1 \leq j \leq q_{n+i})$. 
It is easily seen that $\mathcal{V}_{i, j_1} \subset \mathcal{C}_{k, j_2}$ 
for all $1 \leq i \leq n$, $1 \leq j_1 \leq q_i$, $1 \leq k \leq m$ and 
$1 \leq j_2 \leq q_{n+k}$.

Now, we consider the only part of the construction that depends on which 
literal occurs in which clauses. Denote by $\lambda_{i, k}$ the $k$-th literal 
of clause $c_i$, 
that is write $c_i = \lambda_{i, 1} \vee \lambda_{i, 2} \vee \lambda_{i, 3}$ 
for $1 \leq i \leq m$,
where each $\lambda_{i, k}$ is a variable or its negation. 
The $k$-th literal, $1 \leq k \leq 3$, of each clause $c_i$, $1 \leq i \leq m$, 
is associated to identical subsets $\mathcal{L}_{i, k, j}$, $1 \leq j \leq r_{n+i}$, 
and to identical subsets $\mathcal{L}'_{i, k, j}$, $1 \leq j \leq p_{n+i+1}$. 
These subsets are defined as follows:
$\mathcal{L}_{i, k, j} = \mathbf{V} \cup \mathbf{V}' \cup 
\mathbf{A}_k \cup \mathbf{L}_{i, k} \cup 
\left( \bigcup_{1 \leq \ell \leq i} \mathbf{C}_\ell \right) \cup 
\left( \bigcup_{1 \leq \ell \leq i-1} \mathbf{C}'_\ell \right)$
for $1 \leq i \leq m$, $1 \leq j \leq r_{n+i}$ and $1 \leq k \leq 3$ and
$\mathcal{L}'_{i, k, j} = \mathbf{V} \cup \mathbf{V}' \cup \mathbf{A}_k 
\cup \mathbf{L}_{i, k} \cup \left( \bigcup_{1 \leq \ell \leq i} \mathbf{C}_\ell \right) 
\cup \left( \bigcup_{1 \leq \ell \leq i} \mathbf{C}'_\ell \right)$ 
for $1 \leq i \leq m$, $1 \leq j \leq p_{n+i+1}$ and $1 \leq k \leq 3$, 
where $\mathbf{A}_k = \mathbf{T}_\ell$ if $\lambda_{i, k} = x_\ell$ 
and $\mathbf{A}_k = \mathbf{F}_\ell$ if $\lambda_{i, k} = \overline{x}_\ell$. 
For the sake of clarity, write 
$\mathcal{L}_{i, k} = (\mathcal{L}_{i, k, j} \mid 1 \leq j \leq r_{n+i})$ and 
$\mathcal{L}'_{i, k} = (\mathcal{L}'_{i, k, j} \mid 1 \leq j \leq p_{n+i+1})$ 
for $1 \leq i \leq m$ and $1 \leq k \leq 3$. 
Again, observe that 
$\mathcal{C}_{i, j_1} \subset \mathcal{L}_{i, k, j_2} \subset \mathcal{L}_{i, k, j_2}$ 
for all $1 \leq i \leq m$, $1 \leq j_1 \leq q_{n+i}$, $1 \leq j_2 \leq r_{n+i}$, 
$1 \leq j_3 \leq p_{n+i+1}$ and $1 \leq k \leq 3$.

Our construction ends with $p_1 + K - 1$ \emph{utility subsets}. 
These subsets will be partitioned into two separate classes according to their 
indented function: \emph{bootstrap subsets} and \emph{separator subsets}. 
First, $C$ contains identical bootstrap subsets 
$\mathcal{B}_i$, $1 \leq i \leq p_1 - 1$, defined as follows:
$\mathcal{B}_i = \emptyset$ for $1 \leq i \leq p_1 - 1$. The idea is to force any \SBL\ to map the $p_1 - 1$ 
empty sets of $\mathcal{B}$ to the first $p_1  - 1 = 3(n + m) + 1$ integers. 
Indeed, it is easily seen that all the above defined subsets of the configuration 
of subsets $C$ but those of $\mathcal{B}$ contain at least $p_1$ elements and 
hence cannot be mapped to an integer $i \leq p_1 - 1$ in any \SBL\ of 
$C$. 
Second, $C$ contains identical separator subsets $\mathcal{S}_i$, $1 \leq i \leq K$, 
defined by:
$\mathcal{S}_i = 
\mathbf{V} \cup \mathbf{V}' \cup \mathbf{C} \cup \mathbf{C}' \cup \mathbf{S}$
for $1 \leq i \leq K$.
The rationale of these subsets is that we need a separator between subsets in $C$ 
corresponding to a satisfying truth assignment $f$ for the $3$CNF formula $\phi$ 
and garbage subsets of $C$, that is subsets not involved in the satisfying truth 
assignment $f$. For simplicity, let us denote by $\mathcal{B}$ the collection 
$(\mathcal{B}_i \mid 1 \leq i \leq p_1 - 1)$ and by $\mathcal{S}$ 
the collection $(\mathcal{S}_i \mid 1 \leq i \leq K)$. 
Clearly our construction can be carried on in polynomial time: indeed, we have
$|\mathbf{X}| = O(m^2 + n^2)$ and
$|C| = O(m^2 + n^2)$.

\begin{lemma}
  \label{lemma:Stepwise Bounded Labeling is NPC}
  There exists a satisfying truth assignment $f$ for $\phi$ if and only if 
  there exists an \SBL\ of the configuration of subsets 
  $C$ of the ground set $\mathbf{X}$.
\end{lemma} 

The key elements of the proof are as follows.
First, it is crucial to focus on solutions that map identical subsets 
of elements of $\mathcal{S}$ to a set of consecutive elements
(see Lemma~\ref{lemma:normalized stepwise bounded}).
Second,
the general shape of the solution is largely guided by the construction.
Indeed,
the empty subsets have to be placed first, followed by subsets corresponding to literals 
(either the positive or the negative literal of each variable has been
chosen)
and next by subsets corresponding to clauses
(one satisfying literal of each clause is chose).
Finally the separator subsets have to be placed, with the result that 
(thanks to the large polynomial number of such subsets)
the remaining subsets can be placed in any order without violating
the sought \SBL\ property.
The reader is invited to consider 
Figure~\ref{fig:stepwise bounded labeling construction}
for a schematic illustration of the reduction.
We now briefly discuss, in an informal way, the two key arguments that are used in 
the proof.
First, the whole procedure is, to some extent, similar to 
the accounting method used in amortized complexity analysis. 
Indeed, one might view the operation of placing a set (one after the other)
as the process of charging some customer, the cost being the number of
new elements that are introduced.
With this metaphor in mind, notice that 
we do not charge when a subset does not introduce any new element,
so that the leftover amount can be stored as "credit". 
When we place a new subset that does introduce some new elements, 
we can use the "credit" stored to pay for the cost of the operation.
Second, when a subset uses the "credit" stored to pay the cost
of introducing new elements, the following invariants can be shown 
to hold true:
(i) it uses all the available credit and
(ii) it does not allow to accumulate 
(it should be now clear that consecutive identical subsets do allow
for accumulating credit)
as much credit as it has consumed,
thereby proving that subsets introduce less and less new elements as we progress
adding subsets one after the other.

The main result of this paper can now be stated.

\begin{theorem}
  \label{corollary:PAQ triangular is NPC}
  Let $A$ be a $(0,1)$-matrix.
  Deciding whether $A$ is a 
  \PET\ matrix is \NPC.
\end{theorem}


%%%%%%%%%%%%%%%%%%%%%%%%%%%%%%%%%%%%%%%%%%%%%%%%%%%%%%%%%%%%%%%%%%%%%%%%%%%%%%%

%%
%% ---- Exponential-time algorithm ----
%%
\section{Exponential-time algorithm}
\label{section:Exponential-time algorithm}

We present here an exponential-time algorithm for deciding whether a
given a $(0,1)$-matrix $A$ of order $n$ is a \PET\ matrix.
We start by presenting some basic properties of square $(0,1)$-matrices
that can be transformed into some triangular matrix by row and
column independent permutations to help solving involved 
algorithmic issues. We of course focus of polynomial-time checkable
properties. 

We first focus on the permanent of a square $(0,1)$-matrix.
A well-known result (see e.g. \cite{Brualdi:Ryser:1991})
states that for a $(0,1)$-matrix $A$ of order $n$,
one has $\PER(A)=1$ if and only if the lines of $A$
may be permuted to yield a triangular matrix with $1$'s
in the $n$ main diagonal positions and $0$'s above the mail diagonal.
This theorem amounts to saying that
$\PER(A)=1$ if and only if there exist permutation matrices $P$ and $Q$
such that $I \leq PAQ \leq \lltm$.
As shown in the following lemma, $\PER(A)=1$ is certainly 
a threshold value in our context.

\begin{lemma}
    \label{lemma:permanent}
Let $A$ be $(0,1)$-matrix.
If $A$ is a \PET\  matrix
then $\PER(A) \leq 1$.
\end{lemma}

Notice that deciding $\PER(A) \leq 1$ for $(0,1)$-matrices of order $n$ 
reduces to computing at most
$n+1$ perfect matchings in bipartite graphs 
\cite{Brualdi:Ryser:1991},
and hence the above test is 
$O(n^3\,\sqrt{n})$ time 
as the Hopcroft–Karp algorithm for computing a maximum matching
in a bipartite graph $B = (V,E)$ runs in 
$O(|E|\,\sqrt{|V|})$~\cite{HopcroftKarp:1975}.

Next, it is a simple matter to check that 
if a $(0,1)$-matrix $A$ of order $n$ is a \PET\ matrix, 
then it contains at most $\frac{1}{2}n(n+1)$ $1$'s 
(\emph{i.e.}, $\omega(A) \leq \frac{1}{2}n(n+1)$).
The following lemma gives a lower bound.

\begin{lemma}
    \label{lemma:just a few 1s}
Let $A$ be $(0,1)$-matrix of order $n$, $n \geq 2$.
If $A$ contains at most $n+1$ $1$'s, then $A$ is a \PET\ matrix.
\end{lemma}

Notice that, albeit not very impressive, 
Lemma~\ref{lemma:just a few 1s} is tight as the square matrix
$\begin{bmatrix}
I_{n-2} & \zeros_{n-2,2} \\
\zeros_{2,n-2} & J_{2}
\end{bmatrix}$
of order $n$ has $n-2+4 = n + 2$ $1$'s and is not a \PET\ matrix.

Finally, the following trivial lemma gives another condition
that helps improving the running time of the algorithm in practice.

\begin{lemma}
  \label{lemma:acyclic}
  Let $A$ be $(0,1)$-matrix of order $n$
  and $D$ the directed graph associated to $A$
  (\emph{i.e.}, the adjacency matrix of $D$ is $A$).
  % the $(0,1)$-matrix obtained from $A$
  % by replacing every non-zero diagonal matrix element by zero).
  If the digraph $D$ is acyclic (regardless self-loops),
  then $A$ is a \PET\ matrix.
\end{lemma}

We now turn to presenting the exponential-time algorithm.
The simplest exhaustive algorithm considers every possible pairs
of permutation matrices $(P, Q)$ yielding a $O((n!)^2 \;\cdot\; \POLY(n))$ time
algorithm.
% , where $f(n)$ is the time needed to compute $PAQ$ and check
% $PAQ \leq \lltm_{\,n}$.
However, according to 
Lemma~\ref{lemma:bounded stepwise property <=> PAQ triangular},
it is enough to consider every permutation matrix $P$ of order $n$
and check whether the first $i$, $1 \leq i \leq n$, rows of $PA$
have $1$'s in at most $i$ columns.
This observation yields a $O(n! \,\cdot\, \POLY(n))$ time algorithm.
% where $g(n)$ is the time needed to
% compute $PA$ and check that for the first $i$, $1 \leq i \leq n$,
% rows of $PA$ have $1$'s in at most $i$ columns.
We propose here another exhaustive algorithm that improves on the
$O(n! \,\cdot\, \POLY(n))$ time algorithm. 
The basic idea is to recursively split into smaller submatrices, 
instead of enumerating all permutations.
For a $(0,1)$-matrix $A$ of order $n$, we consider every possible set $R$
of $\left\lceil n/2 \right\rceil$ rows of $A$ and 
every possible set of $\left\lceil n/2 \right\rceil$ columns $C$ of $A$,
and check whether these lines induce a zero matrix
(or a matrix with at most one $1$ in case the matrix has odd order; 
details follow).
%Let $A' = A[R, C]$, that is $A'$ is the submatrix of $A$ with rows $R$ and columns $C$.

\begin{figure}[!t]
    \centering
    \subfigure[Even]{%
$
PAQ =
  \begin{bmatrix}
    A_1 & \zeros \\
    * & A_2
  \end{bmatrix}
$
    \label{fig:mat:even}}
    \qquad
    \subfigure[Odd and one $1$]{%
$
PAQ =
  \begin{bmatrix}
    A_1 & \zeros & \zeros \\
    *       & 1      & \zeros \\
    *       & *      & A_2
  \end{bmatrix}
$
    \label{fig:mat:odd 1}}
    \qquad
    \subfigure[Odd and zero $1$]{%
$
PAQ =
  \begin{bmatrix}
    A_1 & \zeros & \zeros \\
    *       & 0      & \zeros \\
    *       & *      & A_2
  \end{bmatrix}
$
    \label{fig:mat:odd 0}}
    \caption{\label{fig:matrices 1 and 2}}
\end{figure}
If $n$ is even, we let $P$ and $Q$ be two permutation matrices that put the rows in $R$
at the first $\left\lceil n/2 \right\rceil$ positions and the columns in $C$ 
at the last $\left\lceil n/2 \right\rceil$ positions.
The key element for the improvement
is that no specific order is required 
for the rows in $R$ nor for the columns in $C$.
The algorithm rejects the matrix $A$ for the subsets $R$ and $C$ 
if $\omega(A[R, C]) > 1$, otherwise we can write $PAQ$ as in
Figure~\ref{fig:mat:even},
where $A_1$ and $A_2$ are matrices of order $\left\lceil n/2 \right\rceil = n/2$,
and we proceed to recursively check that
both $A_1$ and $A_2$ are \PET\  matrices.
% The algorithm reject the matrix $A$ for the subsets $R$ and $C$ 
% if any of the above-mentioned checks fails, and accept otherwise.

The case $n$ is odd is a little bit more complicated.
%Recall that $A' = A[R,C]$.
First, the algorithm rejects matrix $A$ for the subsets $R$ and $C$ 
if $\omega(A[R, C]) > 1$.
Otherwise,
we need to consider two (possibly positive) cases:
(i) $\omega(A[R, C]) = 1$ 
%(\emph{i.e.}, the submatrix $A[R, C]$ contains exactly one $1$) 
or 
(ii) $\omega(A[R, C]) = 0$.
% (\emph{i.e.}, $A[R, C] = \zeros$).
If $\omega(A[R, C]) = 1$, 
we let $P$ and $Q$ be two permutation matrices that put the rows in $R$
at the first $\left\lceil n/2 \right\rceil$ positions and the columns of $C$ 
at the last $\left\lceil n/2 \right\rceil$ positions
(no specific order for the rows in $R$ nor for the columns in $C$, except that the $1$ of
$A$ is at row index $\left\lceil n/2 \right\rceil$ and at column index $\left\lceil n/2 \right\rceil$
in $PAQ$).
We can write $PAQ$ as in Figure~\ref{fig:mat:odd 1},
where $A_1$ and $A_2$ are matrices of order $\left\lceil n/2 \right\rceil - 1 = n/2$,
and we proceed to recursively check that
both $A_1$ and $A_2$ are \PET\  matrices.
Finally, if $\omega(A[R, C]) = 0$, for every row index $i \in R$ and
every column index $j \in C$,
we let $P$ and $Q$ be two permutation matrices that put the rows in $R$
at the first $\left\lceil n/2 \right\rceil$ positions and the columns of $C$ 
at the last $\left\lceil n/2 \right\rceil$ positions
(no specific order for the rows in $R$ nor for the columns in $C$ except that row $i$ in
$A$ is at row index $\left\lceil n/2 \right\rceil$ and column index $j$ is at
column index $\left\lceil n/2 \right\rceil$ in $PAQ$).
We can write $PAQ$ as in Figure~\ref{fig:mat:odd 0},
where $A_1$ and $A_2$ are matrices of order $\left\lceil n/2 \right\rceil - 1 = n/2$,
and we proceed to recursively check that
both $A_1$ and $A_2$ are \PET\ matrices.

\begin{figure}[!t]
  \begin{algorithm}[H]
    \TitleOfAlgo{\ALGO{permTriangular}}
    \DontPrintSemicolon
    \KwData{A square matrix $A =[a_{i,j}]$ of order $n$}
    \KwResult{true if $A$ is a \PET\ matrix,
    false otherwise}

    \lIf{$(\omega(A) \leq n+1)$ or $(\PER(A) = 1)$ or ($A$ is stepwise bounded) or ($D(A)$ is acyclic)}{
        \Return true
    }
    
    \lIf{$(\omega(A) > \frac{n(n+1)}{2})$ or $(\PER(A) > 1)$ or ($\RSV(A)$ or $\CSV(A)$ is not stepwise bounded)}{
        \Return false
    }

    \For{every subset $R \subset [n]$ of size $\left\lceil\frac{n}{2}\right\rceil$ and
    every subset $C \subset [n]$ of size $\left\lceil\frac{n}{2}\right\rceil$}{
        \eIf{$n$ is even}{
            \Return $\ALGO{permTriangularEven}(A, R, C)$
        }{
            \Return $\ALGO{permTriangularOdd}(A, R, C)$
        }
    }
    \Return false
    \caption{\label{algo:permTriangular}%
    Recognizing \PET\ matrices.}
  \end{algorithm}
\end{figure}


\begin{figure}[!t]
    \begin{algorithm}[H]
    \TitleOfAlgo{\ALGO{permTriangularEven}}
    \DontPrintSemicolon
    \KwData{A square matrix $A =[a_{i,j}]$ of even order $n$, and non-empty subsets
    $R \subset [n]$ and $C \subset [n]$, both of size $\frac{n}{2}$}
    \KwResult{true if $A$ is a \PET\ matrix
    with $A[R,C]$ as the upper right submatrix, false otherwise}
    \lIf{$\omega(A[R, C]) > 0$}{
        % \tcp{$R$ and $C$ do not induce the zero matrix}
        \Return false
    }
    
    Let $A_{\text{ul}} = A[R,\overline{C}]$ and $A_{\text{lr}} = A[\overline{R},C]$ \\
    \Return $\ALGO{permTriangular}(A_{\text{ul}}) \;\&\&\; \ALGO{permTriangular}(A_{\text{lr}})$
    \caption{\label{algo:permTriangularEven}%
    Subprocedure for recognizing \PET\ matrices of even order.}
    \end{algorithm}
\end{figure}


\begin{figure}[!t]
    \begin{algorithm}[H]
    \TitleOfAlgo{\ALGO{permTriangularOdd}}
    \DontPrintSemicolon
    \KwData{A square matrix $A =[a_{i,j}]$ of odd order $n$, and non-empty subsets
     $R \subset [n]$ and $C \subset [n]$, both of size $\left\lceil\frac{n}{2}\right\rceil$}
    \KwResult{true if $A$ is a \PET\ matrix
    with $A[R,C]$ as the upper right submatrix}
    \;
    \lIf{$\omega(A[R, C]) > 1$}{
        % \tcp{$R$ and $C$ do not induce a matrix with at most one $1$}
        \Return false
    }
    
    \eIf{$\omega(A) = 0$}{
        % \tcp{$R$ and $C$ induce a zero matrix}
        \For{every $i \in R$ and every $j \in C$}{
            Let $A_{\text{ul}} = A[R\setminus\{i\},\overline{C}]$ and
            $A_{\text{lr}} = A[\overline{R},C\setminus\{j\}]$ \;
             \lIf{$\ALGO{permTriangular}(A_{\text{ul}}) \;\&\&\;  \ALGO{permTriangular}(A_{\text{lr}})$}{
 %           \If{both $\ALGO{permTriangular}(A_{\text{ul}})$ and $\ALGO{permTriangular}(A_{\text{lr}})$ return true}{
               \Return true
            }
        }
        \Return false
    }{
        % \tcp{$R$ and $C$ induce a matrix with exactly one $1$}
        Let $i$ and $j$ be the row and column indices of the unique $1$ in $A[R, C]$\;
        Let $A_{\text{ul}} = A[R\setminus\{i\},\overline{C}]$ and
        $A_{\text{lr}} = A[\overline{R},C\setminus\{j\}]$ \;
        \Return{$\ALGO{permTriangular}(A_{\text{ul}})) \;\&\&\; \ALGO{permTriangular}(A_{\text{lr}})$}
    }
    \caption{\label{algo:permTriangularOdd}%
    Subprocedure for recognizing \PET\ matrices of odd order.}
    \end{algorithm}
\end{figure}

A detailed description is given in
Algorithms~\ref{algo:permTriangular},
\ref{algo:permTriangularEven} and
\ref{algo:permTriangularOdd}.
We now turn to evaluating the time complexity of this algorithm
and we write $T(n)$ for the time complexity of calling
$\ALGO{permTriangular}(A)$ for some $(0,1)$-matrix $A$ or order $n$.
We have
$$
T(n) \leq
\begin{cases} 
\left(\left\lceil n/2 \right\rceil\right)^2 \binom{n}{\left\lceil n/2 \right\rceil}^2
\left(2T\left(\left\lfloor n/2 \right\rfloor\right)+1\right) 
+ O(n^3\sqrt{n}) & \text{if $n$ is odd} \\
2\,\left(\left\lceil n/2 \right\rceil\right)^2 \binom{n}{\left\lceil n/2 \right\rceil}^2\;
T\left(\left\lfloor n/2 \right\rfloor\right) 
+ O(n^3\sqrt{n}) & \text{if $n$ is even}
\end{cases}
$$
with $T(1) = O(1)$.
The $O(n^3\sqrt{n})$ term
is the time complexity for lines $2$ and $3$ in 
Algorithm~\ref{algo:permTriangular}.
We also observe that the worst case occurs when $n = 2^m - 1$ as 
$\left\lceil n/2 \right\rceil,
\left\lceil n/4 \right\rceil,
\ldots$
are odd integers.
Looking for an asymptotic solution of the worst case, we thus write
the following simplified recurrence:
$
T(2^m)
=
2^{2m - 2} \,
\binom{2^m}{2^{m-1}}^2 \,
\left(2T(2^{m-1}) + 1\right) + 2^{7m/6}
$,
with $T(1) = 1$. 
Now, write $\alpha(2^m) = 2^{2m - 2} \, \binom{2^m}{2^{m-1}}^2$.
Clearly, $\alpha(2^m) \geq 2^{7m/6}$, and hence we focus for now on on
the recurrence
$T(2^m) = 2\,\alpha(2^m)\,\left(T(2^{m-1}) + 1\right)$.
A convenient non-recursive form of $T(2^m)$ is given in the following lemma.

\begin{lemma}
\label{lemma:T}
$
T(2^m)
=
\left(
2^m \prod_{i=1}^{m} \alpha(2^{i})
\right)
+
\left(
\sum_{i=1}^{m}
2^{m-i+1} \prod_{j=i}^{m} \alpha(2^{j})
\right)
$.
\end{lemma}


We now need the following lemma, in order to give an asymptotic
solution for $T(n)$ in Proposition~\ref{proposition:running time}.

% \begin{lemma}
% \label{lemma:prod alpha}
% $$
% \prod_{i=1}^{m} \alpha(2^i)
% =
% O\left(
% \frac{2^{2^{m+2}-3}}{\pi^m}
% \right)
% \text{.}
% $$
% \end{lemma}

\begin{lemma}
\label{lemma:sum prod alpha}
$
\sum_{i=1}^{m}
2^{m-i} \prod_{j=i}^{m} \alpha(2^{j})
=
O\left(m \, 2^{2^{m+2}+m+1}\right)
$.
\end{lemma}




\begin{proposition}
\label{proposition:running time}
Algorithm~\ALGO{permTriangular} runs in 
$O\left(
n\,2^{4n}\,\pi^{-\log(n)}
\right)$ time.
\end{proposition}

\begin{proof}%[of Proposition~\ref{proposition:running time}]
We have already observed that the worst case occurs for $n=2^m-1$.
According to Lemma~\ref{lemma:sum prod alpha},
we have
$
T(2^m) =
O\left(
2^{2^{m+2}+m-3}\,\pi^{-m}
\right)$
and hence
$
T(n) = 
O\left(
2^{2^{\log(n)+2}+\log(n)-3}\,\pi^{-\log(n)}
\right)
=
O\left(
n\,2^{4n}\,\pi^{-\log(n)}
\right)$.
\qed
\end{proof}


%%%%%%%%%%%%%%%%%%%%%%%%%%%%%%%%%%%%%%%%%%%%%%%%%%%%%%%%%%%%%%%%%%%%

%%
%% ---- Conclusion ----
%%
\section{Conclusion}
\label{section:Conclusion}

We suggest for further research directions regarding 
the hardness of recognizing \PET\ $(0,1)$-matrices.
(i)
Suppose a $(0,1)$-matrix $A$ of order $n$ has $n+k$ $1$'s with
$n+2 \leq n+k \leq  \frac{1}{2}n(n+1)$. 
Can one decide in $f(k)\,n^{O(1)}$ time whether 
$A$ is a \PET\ matrix, where $f$ is an arbitrary function depending 
only on $k$?
(ii)
What is the average running time of Algorithm~\ALGO{permTriangular}
for \PET\ matrices?
(iii)
A graph labeling strongly related to symmetric \PET\ $(0,1)$-matrices
can be defined as follows: Given a graph $G=(V,E)$ or order $n$,
decide whether there exists a bijective mapping $f : V \to [n]$ such that
$f(u) + f(v) > n$ for every edge $\{u,v\} \in E$
(\emph{i.e.}, $PAP^T \leq \lrtm_{\,n}$).
Investigating the relationships between the two combinatorial problems
is expected to yield fruitful results.

%%%%%%%%%%%%%%%%%%%%%%%%%%%%%%%%%%%%%%%%%%%%%%%%%%%%%%%%%%%%%%%%%%%%

%%
%% Bibliography
%%

\bibliographystyle{plain}
\bibliography{biblio}

%%%%%%%%%%%%%%%%%%%%%%%%%%%%%%%%%%%%%%%%%%%%%%%%%%%%%%%%%%%%%%%%%%%%%%%%%%%%%%%

\newpage
\section*{Appendix (Reviewers' version only)}

\begin{proof}[of Lemma~\ref{lemma:bounded stepwise property <=> PAQ triangular}]
The forward direction if obvious.
For the reverse direction,
let $\mathcal{S} = (S_i : 1 \leq i \leq n)$ be a configuration of subsets of
some ground set of cardinality $n$
and $\varphi : \mathcal{S} \to [n]$ be a bijective mapping
such that $\left|\bigcup_{S \in \mathcal{S} : \varphi(S) \leq i} S\right| \leq i$
for all $1\leq i \leq n$.
Let $A = [a_{i,j}]$ be the incidence matrix of $\mathcal{S}$.
Now let $P$ be the permutation matrix of order $n$ that permutes
the rows of $A$ so that row $i$ of $PA$ correspond to subset
$\varphi^{-1}(i)$ for all $1 \leq i \leq n$.
We denote this row-permuted matrix by $A' = [a'_{i,j}]$.
Define the function $\tau : [n] \to [n+1]$ that indicates for each column index
$j$ the minimum row index $i$ so that $a'_{i,j} = 1$, and we adhere to the convention
that $\tau(j) = n+1$ if row $j$ does not contain a $1$.
Define a permutation matrix $Q$ or order $n$ that permutes the
columns of $A'$ by ascending $\tau$ values,
breaking ties arbitrarily.
We designate this row-permuted matrix by $A'' = [a''_{i,j}]$.
We claim that $A'' = PAQ \leq \lltm_{\,n}$.
Indeed, suppose, aiming at a contradiction, that
$A'' \not\leq \lltm_{\,n}$.
Let $i_{\min}$ be the smallest row index such that
$a''_{i_{\min},j_{\min}} = 1$
for some $j_{\min} > i_{\min}$.
Then it follows that $\tau(j) \leq \tau(j_{\min})$ for $j < j_{\min}$
and hence
$\left|\bigcup_{S \in \mathcal{S} : \varphi(S) \leq i_{\min}} S\right| > i_{\min}$.
This is the desired contradiction, and hence
$A'' = PAQ \leq \lltm_{\,n}$.
\qed
\end{proof}

\horiz

\begin{proof}[of Lemma~\ref{lemma:normalized stepwise bounded}]
  The proof is by contradiction. Denote by $\Phi(\mathcal{S})$ the set
  of all \SBL\ s of the configuration of subsets $\mathcal{S}$. We claim that
  there exists a normalized \SBL\ $\varphi \in \Phi(\mathcal{S})$
  which maps the identical subsets of $\mathcal{S}$ to sets of consecutive
  integers. For each $\varphi \in \Phi(\mathcal{S})$, define
  $$
  \mathcal{M}(\varphi) = \{S_j \mid \exists S_i, S_k \text{ s.t. }
  \varphi(S_i) < \varphi(S_k) < \varphi(S_j) \text{ and } S_i = S_j \neq S_k\}
  $$
  Then, there exists a mapping $\varphi^* \in \Phi(\mathcal{S})$ such that
  $|\mathcal{M}(\varphi^*)| \leq |\mathcal{M}(\varphi')|$ for all $\varphi' \in
  \Phi(\mathcal{S})$. We show that $|\mathcal{M}(\varphi^*)| = 0$, and hence that
  $\varphi^*$ is our desired mapping. Suppose, for the sake of contradiction,
  that $|\mathcal{M}(\varphi^*)| > 0$. Let $S_j \in \mathcal{M}(\varphi^*)$ be
  such that $\varphi^*(S_j) \leq \varphi^*(S)$ for all
  $S \in \mathcal{M}(\varphi^*)$. Then, there exists two subsets $S_i, S_k \in \mathcal{S}$
  such that $S_i = S_j \neq S_k$ and
  $\varphi^*(S_i) + 1 = \varphi^*(S_k) < \varphi^*(S_j)$.
  Consider a new labeling $\varphi'$ defined by
  \begin{alignat*}{2}
    &\varphi'(S) = \varphi^*(S)         &\qquad& \text{for all $S \in \mathcal{S}$ such that
      $1 \leq \varphi^*(S) \leq \varphi^*(S_i)$} \\
    &\varphi'(S) = \varphi^*(S) + 1     &\qquad& \text{for all $S \in \mathcal{S}$ such that
      $\varphi^*(S_i) < \varphi^*(S) < \varphi^*(S_j)$} \\
    &\varphi'(S_j) = \varphi^*(S_k)               &\qquad& \\
    &\varphi'(S) = \varphi^*(S) + 1     &\qquad& \text{for all $S \in \mathcal{S}$ such that
      $\varphi^*(S_j) < \varphi^*(S) \leq m$}
  \end{alignat*}

%   \begin{minipage}[c]{\textwidth}
%     \bigskip
%     \centering
%     \input{fig/fig-normalized-layout.pstex_t}
%     \bigskip
%   \end{minipage}

  For simplicity of notation, we write $i$ instead of $\varphi^*(S_i)$,
  $j$ instead of $\varphi^*(S_j)$ and $k$ instead of
  $\varphi^*(S_k)$. Observe that $i+1 = k$. We claim that $\varphi'$ is
  an \SBL\ for $\mathcal{S}$. It is sufficient to show that 
  \begin{align*}
    \bigcup_{\varphi'(S) \leq \ell} S &\subseteq \bigcup_{\varphi^*(S) \leq \ell} S \\
    \intertext{for all $k \leq \ell \leq j$. We check at once that}
    \bigcup_{1 \leq \varphi'(S) \leq \ell} S &= \left( \bigcup_{1 \leq \varphi'(S) \leq i} S \right) \;\cup\; S_j \;\cup\; \left( \bigcup_{k+1 \leq \varphi'(S) \leq \ell} S \right) \\
    &= \left( \bigcup_{1 \leq \varphi^*(S) \leq i} S \right) \;\cup\; S_j \;\cup\; \left( \bigcup_{k \leq \varphi^*(S) \leq \ell-1} S \right) \\
    &= \left( \bigcup_{1 \leq \varphi^*(S) \leq \ell-1} S \right) \;\cup\; S_j \\
    &= \bigcup_{1 \leq \varphi^*(S) \leq \ell-1} S \qquad\qquad \text{(since $S_i = S_j$)} \\
    &\subseteq \bigcup_{1 \leq \varphi^*(S) \leq \ell} S
  \end{align*}
  It follows immediately that $S_j \notin \mathcal{M}(\varphi')$. Indeed, $\varphi^*(S_j) \leq \varphi^*(S)$ for all $S \in \mathcal{M}(\varphi^*)$.

  We proceed to show that $|\mathcal{M}(\varphi^*)| >
  |\mathcal{M}(\varphi')|$. If $|\mathcal{M}(\varphi')| = 0$, we are done, so
  that we may assume $|\mathcal{M}(\varphi')| > 0$. Let $S' \in
  \mathcal{M}(\varphi')$. Then there exist subsets $S_a$ and $S_b$ such that
  $\varphi'(S_a) < \varphi'(S_b) < \varphi'(S')$ and $S_a = S' \neq S_b$. Now,
  observe that we must have $\varphi^*(S_j) < \varphi^*(S')$. Again, this follows
  from the fact that $\varphi^*(S_j) \leq \varphi^*(S)$ for all $S \in
  \mathcal{M}(\varphi^*)$. Therefore, by construction of the labeling
  $\varphi'$, either $\varphi^*(S_a) < \varphi^*(S_b) < \varphi^*(S')$ or $\varphi^*(S_b) <
  \varphi^*(S_a) < \varphi^*(S')$. We claim that $S' \in \mathcal{M}(\varphi^*)$. The
  result is certainly valid in case $\varphi^*(S_a) < \varphi^*(S_b)$. Suppose now
  that $\varphi^*(S_b) < \varphi^*(S_a)$. Then it follows that we must have $S_a =
  S_j$. But $S_i = S_j$. Therefore $\varphi^*(S_i) < \varphi^*(S_b) < \varphi^*(S')$
  and $S_i = S' \neq S_b$, and hence $S' \in
  \mathcal{M}(\varphi^*)$. Combining this with the fact that $S_j \notin
  \mathcal{M}(\varphi')$ yields  $|\mathcal{M}(\varphi^*)| >
  |\mathcal{M}(\varphi')|$. This contradicts the choice of
  $\varphi^*$. Therefore, we must have $|\mathcal{M}(\varphi^*)| = 0$ and hence
  $\varphi^*$ is a normalized \SBL\ of $\mathcal{S}$.
  \qed
\end{proof}

\horiz

\begin{proof}[of Lemma~\ref{lemma:Stepwise Bounded Labeling is NPC}]
We claim that there exists a satisfying truth assignment $f$ for
$\phi$ if and only if there exists an \SBL\ of the configuration of subsets $C$ of the ground set $\mathbf{X}$.

  Suppose that there exists a satisfying truth assignment $f$ for the 
  $3$CNF formula $\phi$. 
  Due to symmetry, there is no loss of generality in assuming that each clause is 
  satisfied by its first literal. 
  Define an labeling $\tau$ of the configuration of subsets $C$ as follows. 
  First, we begin our labeling construction by placing all the empty subsets of $C$, that is
  $$
  1 \leq \tau(\mathcal{B}_i) \leq p_1 - 1
  $$
  for all $\mathcal{B}_i \in B$. 
  Next, all the subsets corresponding to the variables are ordered as follows: 
  for all $\mathcal{V}_{i, j} \in \mathcal{V}_{i}$, $1 \leq i \leq n$,
  $$
  p_i + \sum_{k = 1}^{i-1} (p_k + q_k + r_k) \leq \tau(\mathcal{V}_{i, j}) \leq p_i + q_i + \sum_{k = 1}^{i-1} (p_k + q_k + r_k) - 1
  $$
  The satisfying truth assignment $f$ for $\phi$ is coded in our construction as follows: 
  for all $\mathcal{A}_{i, j} \in \mathcal{A}_i$, $1 \leq i \leq n$, 
  where $\mathcal{A}_i = \mathcal{T}_i$ if $f(x_i) = \mbox{TRUE}$ and 
  $\mathcal{A}_i = \mathcal{F}_i$ if $f(x_i) = \mbox{FALSE}$,
  $$
  p_i + q_i + \sum_{k = 1}^{i-1} (p_k + q_k + r_k) \leq \tau(\mathcal{A}_{i, j}) \leq \sum_{k = 1}^{i} (p_k + q_k + r_k) - 1
  $$
  and for all $\mathcal{A}'_{i, j} \in \mathcal{A}'_i$, $1 \leq i \leq n$, 
  where $\mathcal{A}'_i = \mathcal{T}'_i$ if $f(x_i) = \mbox{TRUE}$ and $\mathcal{A}'_i = \mathcal{F}'_i$ if $f(x_i) = \mbox{FALSE}$,
  $$
  \sum_{k = 1}^{i} (p_k + q_k + r_k) \leq \tau(\mathcal{A}'_{i, j}) \leq p_{i+1} + \sum_{k = 1}^{i} (p_k + q_k + r_k) - 1
  $$
  It easily follows that
  $$
  \tau(\mathcal{V}_i) < \tau(\mathcal{A}_i) < \tau(\mathcal{A}'_i) < \tau(\mathcal{V}_j) < \tau(\mathcal{A}_j) < \tau(\mathcal{A}'_j)
  $$
  for all $1 \leq i < j \leq n$.

  Having disposed of all those subsets corresponding to the variables and to the satisfying truth assignment $f$ for $\phi$, we now turn to the clauses. This will be divided into two parts. First, the subsets of $\mathcal{C}_i$, $1 \leq i \leq m$, are ordered as follows: for all $\mathcal{C}_{i, j} \in \mathcal{C}_i$, $1 \leq i \leq m$,
  $$
  p_{n+i} + \sum_{k = 1}^{n+i-1} (p_k + q_k + r_k) \leq \tau(\mathcal{C}_{i, j}) \leq p_{n+i} + q_{n+i} + \sum_{k = 1}^{n+i-1} (p_k + q_k + r_k) - 1
  $$
  Second, all the subsets of $\mathcal{L}_{i, 1}$ and $\mathcal{L}'_{i, 1}$, $1 \leq i \leq m$, are ordered as follows\footnote{Recall that we assume that each clause is satisfied by its first literal.}: for all $\mathcal{L}_{i, 1, j} \in \mathcal{L}_{i, 1}$, $1 \leq i \leq m$,
  $$
  p_{n+i} + q_{n+i} + \sum_{k = 1}^{n+i-1} (p_k + q_k + r_k) \leq \tau(\mathcal{L}_{i, 1, j}) \leq \sum_{k = 1}^{n+i} (p_k + q_k + r_k) - 1
  $$
  and for all $\mathcal{L}'_{i, 1, j} \in \mathcal{L}'_{i, 1}$, $1 \leq i \leq m$,
  $$
  \sum_{k = 1}^{n+i} (p_k + q_k + r_k) \leq \tau(\mathcal{L}'_{i, 1, j}) \leq p_{n+i+1} + \sum_{k = 1}^{n+i} (p_k + q_k + r_k) - 1
  $$
  A trivial verification shows that
  $$
  \tau(\mathcal{C}_i) < \tau(\mathcal{L}_{i,1}) < \tau(\mathcal{L}'_{i, 1}) < \tau(\mathcal{C}_j) < \tau(\mathcal{L}_{j,1}) < \tau(\mathcal{L}'_{j, 1})
  $$
  for all $1 \leq i < j \leq m$.

  Here come the $K$ separator subsets of $\mathcal{S}$, that is for all $\mathcal{S}_i \in \mathcal{S}$
  $$
  p_{n+m+1} + \sum_{k = 1}^{n+m} (p_k + q_k + r_k) \leq \tau(\mathcal{S}_i) \leq p_{n+m+1} + K + \sum_{k = 1}^{n+m} (p_k + q_k + r_k) - 1
  $$
  Up to now, all the subsets of $C$ but $L$ of them have been used to define $\tau$. Our labeling construction ends with those subsets by packing them in the $L$ last places:
  $$
  p_{n+m+1} + K + \sum_{k = 1}^{n+m} (p_k + q_k + r_k) \leq \tau(\mathcal{X}_j) \leq p_{n+m+1} + K + L + \sum_{k = 1}^{n+m} (p_k + q_k + r_k) - 1
  $$
  A careful examination of $\tau$ shows that $\left| \bigcup_{\tau(X)
    \leq i} X \right| \leq i$ for $1 \leq i \leq |C|$ and hence that
  $\tau$ is an \SBL\footnote{Observe that the construction of $\tau$
    may be specialized to yield a normalized \SBL. Indeed, all the subsets of $C$ but the $L$ last ones are
    mapped by $\tau$ to a set of consecutive integers. But there is no
    less of generality in assuming that these subsets are mapped by
    $\tau$ to a set of consecutive integers as well.} of the
  configuration of subsets $C$ of the ground set $\mathbf{X}$. An
  (partial) illustration of the construction of the \SBL\ $\tau$ is shown in Figure~\ref{fig:stepwise bounded
    labeling construction}. 

  % \begin{figure}[ht]
  %   \centering
  %   \bigskip
  %   \begin{minipage}[c]{\textwidth}
  %     \centering
  %     \input{r.pstex_t}
  %     \caption{\label{fig:stepwise bounded labeling
  %         construction}Illustration of the construction used in the
  %       proof of proposition~\ref{proposition:Stepwise Bounded
  %         Labeling is NPC} for the $3$CNF formula $\phi = (x_1 \vee
  %       x_2 \vee x_3) \wedge (\overline{x}_1 \vee \overline{x}_2
  %       \vee x_3) \wedge (\overline{x}_1 \vee x_2 \vee
  %       \overline{x}_3)$. Identical subsets are not distinguishable
  %       in our representation. A satisfying truth assignment is
  %       given by $f(x_1) = \mbox{TRUE}$, $f(x_2) = \mbox{FALSE}$ and
  %       $f(x_3) = \mbox{FALSE}$. For the sake of clarity, neither
  %       the ground set $\mathbf{X}$ nor the collection of subsets
  %       $C$ is fully represented. Note that we have here dropped the
  %       assumption that each clause is satisfied by its first
  %       literal.} 
  %   \end{minipage}%

  %   \bigskip
  % \end{figure}

  For the converse, suppose that there exists an \SBL\ $\tau$ for $C$, that is an labeling such that $\left|
    \bigcup_{\tau(X_j) \leq i} X_j \right| \leq i$ for $1 \leq i \leq
  |C|$. According to lemma~\ref{lemma:normalized stepwise bounded}, there is
  no loss of generality in assuming that $\tau$ is a normalized \SBL\, \emph{i.e.}, the identical subsets of the
  configuration $C$ are mapped by $\tau$ to a set of consecutive
  integers. This property is crucial in our  proof.  

  Let us start by proving that $1 \leq \tau(\mathcal{B}_i) \leq p_1 - 1$
  for all $\mathcal{B}_i \in \mathcal{B}$. Indeed, as mentioned earlier
  in the proof, all the above defined subsets of the configuration of
  subsets $C$ but those of $\mathcal{B}$ contain at least $p_1$ elements
  and hence can not be mapped to an integer $i \leq p_1 - 1$ in any
  \SBL\ $\tau$ for $C$. 

  Define the subcollection $C^* \subset C$ as follows:
  $$
  C^* = (X \in C \mid \max \{\tau(\mathcal{B}_i) \mid \mathcal{B}_i \in
  \mathcal{B}\} < \tau(X) < \min \{\tau(\mathcal{S}_i) \mid
  \mathcal{S}_i \in \mathcal{S}\}) 
  $$
  In other words, the subcollection $C^*$ contains all those subsets
  that go after the last bootstrap subset $\mathcal{B}_i \in
  \mathcal{B}$ and before the first separator subset $\mathcal{S}_i \in
  \mathcal{S}$. We claim that $C^*$ contains (1) the subsets of
  $\mathcal{V}_i$ for $1 \leq i \leq n$, (2) either the subsets of
  $\mathcal{T}_i$ and $\mathcal{T}'_i$ or the subsets of $\mathcal{F}_i$
  and $\mathcal{F}'_i$ for $1 \leq i \leq n$, (3) the subsets of
  $\mathcal{C}_i$ for $1 \leq i \leq m$ and (4) the subsets of
  $\mathcal{L}_{i,1}$ and $\mathcal{L}'_{i,1}$ or the subsets of
  $\mathcal{L}_{i,2}$ and $\mathcal{L}'_{i,2}$ or the subsets of
  $\mathcal{L}_{i,3}$ and $\mathcal{L}'_{i,3}$ for $1 \leq i \leq
  m$. For simplicity, this will be divided into several steps. 

  Having disposed of the bootstrap subsets, we now turn to the non-empty
  subsets of the configuration of subsets $C$. Roughly speaking, we show
  that our construction implies a \emph{force placement} of the subsets
  of $C$ together with choices corresponding to a satisfying truth
  assignment for the $3$CNF formula $\phi$. First, we must have: 
  $$
  p_1 \leq \tau(\mathcal{V}_{1, j}) \leq p_1 + q_1 - 1
  $$
  for all $1 \leq j \leq q_1$. Indeed, all other subsets have more than
  $p_1$ elements and hence can not be mapped to $p_1$ by
  $\tau$. Moreover, $\tau$ is a normalized \SBL\ for
  $C$, \emph{i.e.}, the subsets $\mathcal{V}_{1, j}$, $1 \leq j \leq
  q_1$, are mapped by $\tau$ to a set of consecutive integers. Now, the
  key point is that the subset which is mapped to $p_1 + q_1$ by $\tau$
  must have cardinality at most $p_1 + q_1$. But a careful examination
  of the configuration of subsets $C$ shows that all the remaining
  subsets contain $\mathbf{V}_1$, and hence the subset which is mapped
  to $p_1 + q_1$ by $\tau$ may introduce at most $q_1$ new elements of
  the ground set. Then it follows from our construction that this subset
  is either a subset of the collection $\mathcal{T}_1$ or a subset of
  the collection $\mathcal{F}_1$. Indeed, it is sufficient to note that
  $r_1 + p_2 > q_1$, and hence that no subset $\mathcal{V}_{2, j} \in
  \mathcal{V}_2$ can satisfy $\tau(\mathcal{V}_{2, j}) = p_1 + q_1$. As
  an immediate result, exactly one of the following two statements is
  true: 
  \begin{align}
    &p_1 + q_1 \leq \tau(\mathcal{T}_{1, j}) \leq p_1 + q_1 + r_1 - 1
    \label{eq:stepwise bounded labeling 1-T} \\ 
    &p_1 + q_1 \leq \tau(\mathcal{F}_{1, j}) \leq p_1 + q_1 + r_1 - 1
    \label{eq:stepwise bounded labeling 1-F} 
  \end{align}
  for all $1 \leq j \leq r_1$. We can now proceed analogously to the
  above to obtain: 
  \begin{align*}
    \eqref{eq:stepwise bounded labeling 1-T} &\quad\Rightarrow\quad p_1 +
    q_1 + r_1 \leq \tau(\mathcal{T}'_{1, j}) \leq p_1 + q_1 + r_1 +p_2 - 1
    \\ 
    \eqref{eq:stepwise bounded labeling 1-F} &\quad\Rightarrow\quad p_1 +
    q_1 + r_1 \leq \tau(\mathcal{F}'_{1, j}) \leq p_1 + q_1 + r_1 +p_2 - 1 
  \end{align*}
  for all $1 \leq j \leq p_2$. Summarizing, exactly one of the following
  two statements is true: 
  \begin{align*}
    p_1 \leq \tau(\mathcal{V}_1) < \tau(\mathcal{T}_1) < \tau(\mathcal{T}'_1) \leq p_1 + q_1 + r_1 +p_2 - 1 \\
    p_1 \leq \tau(\mathcal{V}_1) < \tau(\mathcal{F}_1) < \tau(\mathcal{F}'_1) \leq p_1 + q_1 + r_1 +p_2 - 1
  \end{align*}
  We continue in this fashion obtaining that exactly one of the following two statements is true:
  \begin{align*}
    &p_i + \sum_{k=1}^{i-1} (p_k + q_k + r_k) \leq \tau(\mathcal{V}_i) < \tau(\mathcal{T}_i) < \tau(\mathcal{T}'_i) \leq p_{i+1} + \sum_{k=1}^{i} (p_k + q_k + r_k) - 1 \\
    &p_i + \sum_{k=1}^{i-1} (p_k + q_k + r_k) \leq \tau(\mathcal{V}_i) < \tau(\mathcal{F}_i) < \tau(\mathcal{F}'_i) \leq p_{i+1} + \sum_{k=1}^{i} (p_k + q_k + r_k) - 1 
  \end{align*}
  for all $1 \leq i \leq n$. This result is crucial as it allows us to construct a truth assignment $f$ for the $3$CNF formula $\phi$. Indeed, subsets of $\mathcal{T}_i$ and $\mathcal{T}'_i$ are associated in our construction to literal $x_i$ while subsets of $\mathcal{F}_i$ and $\mathcal{F}'_i$ are associated in our construction to literal $\overline{x}_i$.

  Having disposed of the variables we now turn to the clauses. For the sake of clarity, let us first introduce the temporary notations
  \begin{align*}
    l_i &= p_{n+i} + \sum_{k=1}^{n+i-1} (p_k + q_k + r_k) \\
    h_i  &= p_{n+i+1} + \sum_{k=1}^{n+i} (p_k + q_k + r_k) - 1
  \end{align*}
  for $1 \leq i \leq m$. We may now proceed analogously to the above to obtain that 
  exactly one of the following three statements is true:
  \begin{align*}
    &l_i \leq \tau(\mathcal{C}_i) < \tau(\mathcal{L}_{i, 1}) < \tau(\mathcal{L}'_{i, 1}) \leq h_i \\
    &l_i \leq \tau(\mathcal{C}_i) < \tau(\mathcal{L}_{i, 2}) < \tau(\mathcal{L}'_{i, 2}) \leq h_i \\
    &l_i \leq \tau(\mathcal{C}_i) < \tau(\mathcal{L}_{i, 3}) < \tau(\mathcal{L}'_{i, 3}) \leq h_i 
  \end{align*}
  for all $1 \leq i \leq m$. 

  According to the above, 
  the subcollection $C^*$ contains either the subsets of $\mathcal{T}_i$ and $\mathcal{T}'_i$ 
  or the subsets of $\mathcal{F}_i$ and $\mathcal{F}'_i$ for $1 \leq i \leq n$. 
  Therefore we can define a truth assignment $f$ for the $3$CNF formula $\phi$ as follows: 
  $f(x_i) = \mbox{TRUE}$ if $\mathcal{T}_i \subset C^*$ and $f(x_i) = \mbox{FALSE}$ 
  if $\mathcal{F}_i \subset C^*$ for $1 \leq i \leq n$. 
  We claim that $f$ is a satisfying truth assignment for $\phi$. 
  Indeed, for each $i$, $1 \leq i \leq m$, consider the true statement
  $$
  l_i \leq \tau(\mathcal{C}_i) < \tau(\mathcal{L}_{i, k}) < \tau(\mathcal{L}'_{i, k}) \leq h_i
  $$
  where $k$ is either $1$, $2$ or $3$. By construction we have
  \begin{align*}
    \mathcal{L}_{i, k, j}  &= \mathbf{V} \cup \mathbf{V}' \cup \mathbf{A}_k \cup \mathbf{L}_{i, k} \cup \left( \bigcup_{1 \leq \ell \leq i} \mathbf{C}_\ell \right) \cup \left( \bigcup_{1 \leq \ell \leq i-1} \mathbf{C}'_\ell \right) \\
    \intertext{for $1 \leq j \leq r_{n+i}$ and}
    \mathcal{L}'_{i, k, j} &= \mathbf{V} \cup \mathbf{V}' \cup \mathbf{A}_k \cup \mathbf{L}_{i, k} \cup \left( \bigcup_{1 \leq \ell \leq i} \mathbf{C}_\ell \right) \cup \left( \bigcup_{1 \leq \ell \leq i} \mathbf{C}'_\ell \right) 
  \end{align*}
  for $1 \leq j \leq p_{n+i+1}$, where $\mathbf{A}_k = \mathbf{T}_\ell$
  if $\lambda_{i, k} = x_\ell$ and $\mathbf{A}_k = \mathbf{F}_\ell$ if
  $\lambda_{i, k} = \overline{x}_\ell$. But a careful examination of the
  configuration of subsets $C$ and of the \SBL\
  $\tau$ shows that none of the previous subsets contain the elements of
  $\mathbf{L}_{i, k}$. Since $|\mathcal{C}_i| = q_{n+i}$ and
  $|\mathbf{L}_{i, k}| = q_{n+i}$, then it follows that all the elements
  of $\mathbf{A}_k$ must have been introduced by previous
  subsets. Indeed, $l_i \leq \tau(\mathcal{C}_i) < \tau(\mathcal{L}_{i,
    k}) < \tau(\mathcal{L}'_{i, k}) \leq h_i$ and hence $\mathcal{C}_i$
  and $\mathcal{L}_{i, k}$ are mapped by $\tau$ to a set of consecutive
  integers. Therefore, if $\mathbf{A}_k = \mathbf{T}_\ell$
  (resp. $\mathbf{A}_k = \mathbf{F}_\ell$) then we must have chosen
  $\mathcal{T}_\ell$ (resp. $\mathcal{F}_\ell$) in a previous step, that
  is $\mathcal{T}_\ell, \mathcal{T}'_\ell \in C^*$
  (resp. $\mathcal{F}_\ell, \mathcal{F}'_\ell \in C^*$). Then it follows
  that clause $c_i$ is satisfied by its $k$-th literal. Hence, $f$ is a
  satisfying truth assignment for $\phi$ and the lemma is proved. 
\qed
\end{proof}

\horiz

\begin{proof}[of Lemma~\ref{lemma:permanent}]
The proof is by induction.
The result is obvious for $n=1$.
For the inductive step, assume the statement holds for some natural number $n$
and let $A$ be a $(0,1)$-matrix of order $n+1$.
Since $A$ is assumed to be a \PET\
matrix there exist permutation matrices $P$ and $Q$ such that
$PAQ \leq \lltm_{\,n}$.
Let $A' = PAQ$.
As $\PER(A)$ remains invariant under arbitrary permutation of the lines of $A$
we have $\PER(A) = \PER(A')$.
If $\PER(A') = 0$, we are done.
Otherwise $\PER(A') \geq 1$ and hence
$n$ $1$'s appear on the main diagonal of $A'$.
We designate the matrix obtained from $A'$ by deleting
the first row and the first column permuted matrix by $A''$.
As the first row of $A'$ contains exactly one $1$,
we have $\PER(A'') = \PER(A') = \PER(A) \leq 1$, thereby proving the lemma.
\qed
\end{proof}

\horiz

\begin{proof}[of Lemma \ref{lemma:just a few 1s}]
The proof is by induction.
The assertion is certainly valid for $n = 2$ as
all $(0,1)$-matrix of order $2$ are \PET\ matrices but $J_2$.
For the inductive step, assume the statement holds for some natural number $n$ and let
$A$ be $(0,1)$-matrix of order $n+1$.
If $A = \zeros$ we are done.
Otherwise, let $\alpha_i$ be a row of $A$ with minimum row sum $r_i$.
Since $A$ contains at most $n+1$ $1$'s, we certainly have $r_i \leq 1$.
We need to consider two cases:
\begin{itemize}
\item
If $r_i = 0$,
let $\beta_j$ be a column of $A$ with maximum column sum $c_j$.
Since $A \neq \zeros$, we certainly have $c_j > 0$.
We now permute the lines of $A$ so that row $\alpha_i$ and
column $\beta_j$ are the first row and column of the permuted matrix.
We now delete the first row and column of this permuted matrix and
apply the induction hypothesis to this submatrix of order $n$.
\item
If $r_i = 1$,
we may permute the rows of $A$ so that row $1$ of the permuted matrix
contains a $1$ in the $(1,1)$ position and $O$'s elsewhere.
We now delete the first row and column of this matrix and apply the induction
hypothesis to this submatrix of order $n$.
\end{itemize}
\qed
\end{proof}

\horiz

\begin{proof}[of Lemma~\ref{lemma:T}]
The proof is by induction on $m$.
First, the assertion is certainly valid for $m = 1$ as
\begin{align*}
2\,\alpha(2^1)\,(T(2^0) + 1)
&= 
2\times 4 \times (1 + 1) 
\\
&=
16
\\
\intertext{and}
\left(
2^1 \prod_{i=1}^{1} \alpha(2^{i})
\right)
+
\left(
\sum_{i=1}^{1}
2^{1-i+1} \prod_{j=i}^{1} \alpha(2^{j})
\right)
&= (2^1 \times \alpha(2^1)) + (2^1 \times \alpha(2^1))
\\
&=
(2 \times 4) + (2 \times 4) 
\\
&=
16\text{.}
\end{align*}

Suppose now that the assertion is true for $2^m$.
Then
\begin{align*}
T(2^{m+1})
&=
2\,\alpha(2^{m+1})\,T(2^{m}) + 2\,\alpha(2^{m+1})
\\
&=
2\,\alpha(2^{m+1})\,
\left(
\left(
2^m \prod_{i=1}^{m} \alpha(2^{i})
\right)
+
\left(
\sum_{i=1}^{m}
2^{m-i+1} \prod_{j=i}^{m} \alpha(2^{j})
\right)
\right)
 + 2\,\alpha(2^{m+1})
 \\
&=
\left(
2^{m+1} \prod_{i=1}^{m+1} \alpha(2^{i})
\right)
+
\left(
\sum_{i=1}^{m}
2^{m+1-i+1} \prod_{j=i}^{m+1} \alpha(2^{j})
\right)
 + 2\,\alpha(2^{m+1})
 \\ 
&=
\left(
2^{m+1} \prod_{i=1}^{m+1} \alpha(2^{i})
\right)
+
\left(
\sum_{i=1}^{m+1}
2^{m+1-i+1} \prod_{j=i}^{m+1} \alpha(2^{j})
\right)
- 2\,\alpha(2^{m+1}) 
+ 2\,\alpha(2^{m+1})
 \\
&=
\left(
2^{m+1} \prod_{i=1}^{m+1} \alpha(2^{i})
\right)
+
\left(
\sum_{i=1}^{m+1}
2^{m+1-i+1} \prod_{j=i}^{m+1} \alpha(2^{j})
\right)\text{.} 
\end{align*}
\qed
\end{proof}

\horiz

\begin{proof}[of Lemma~\ref{lemma:sum prod alpha}]
We need the following claims.

\begin{claim}
\label{claim:inner:1}
$$
\alpha(2^i)
=
\frac{2^{2^{m+1} + m - 1}}{\pi}\left(1 + O\left(2^{1-m}\right)\right)
\text{.}
$$
\end{claim}
\begin{proof}
We use the following well-known asymptotic (see e.g. \cite{ConcreteMathematics:1994}):
\begin{align*}
\binom{2n}{n} &= 
\frac{4^n}{\sqrt{\pi n}} \left(1 + O\left(n^{-1}\right)\right)\text{.}
\\
\intertext{Therefore,}
\alpha(2^m)
&=
2^{2m-2} \, \binom{2^m}{2^{m-1}}^2
\\
&=
2^{2m-2}\;
\left(\frac{4^{2^{m-1}}}{\sqrt{\pi 2^{m-1}}} \left(1 + O\left(2^{1-m}\right)\right)\right)^2
\\
&=
2^{2m-2}\;
\left(\frac{4^{2^{m-1}}}{\sqrt{\pi 2^{m-1}}}\right)^2 \left(1 + O\left(2^{1-m}\right)\right)^2
\\
&=
2^{2m-2}\;
\frac{4^{2^m}}{\pi 2^{m-1}} \left(1 + O\left(2^{1-m}\right)\right)^2
\\
&=
\frac{2^{2^{m+1} + m - 1}}{\pi}\left(1 + O\left(2^{1-m}\right)\right)^2
\text{.}
\\
\intertext{We now develop the error term to obtain}
\alpha(2^m)
&=
\frac{2^{2^{m+1} + m - 1}}{\pi}\,
\left(1 + O\left(2^{1-m}\right)\right)
\,
\left(1 + O\left(2^{1-m}\right)\right)
\\
&=
\frac{2^{2^{m+1} + m - 1}}{\pi}\,
\left(
1 + O\left(2^{1-m}\right) + O\left(2^{1-m}\right) + O\left(2^{2-2m}\right)
\right)
\\
&=
\frac{2^{2^{m+1} + m - 1}}{\pi}\left(1 + O\left(2^{1-m}\right)\right)
\text{.}
\end{align*}
\qed
\end{proof}

\begin{claim}
\label{claim:inner:2}
$$
\prod_{i=1}^{m} \alpha(2^i)
=
O\left(
\frac{2^{2^{m+2}-3}}{\pi^m}
\right)
\text{.}
$$
\end{claim}
\begin{proof}
According to Claim~\ref{claim:inner:1}, we may write
\begin{align*}
\prod_{i=1}^{m} \alpha(2^i)
&=
\prod_{i=1}^{m} \frac{2^{2^{i+1} + i - 1}}{\pi}\left(1 + O\left(2^{1-i}\right)\right)
\\
&=
\left(
\frac{1}{2\pi}
\right)^{m}
\,
\left(
\prod_{i=1}^{m} 2^{2^{i+1}}
\right)
\,
\left(
\prod_{i=1}^{m} 2^i
\right)
\,
\left(
\prod_{i=1}^{m} \left(1 + O\left(2^{1-i}\right)\right)
\right)
\\
&=
\left(
\frac{1}{2\pi}
\right)^{m}
\,
\left(2^{2^{m+2}-4}\right)
\,
\left(2^{m+1}-2\right)
\,
\left(
\prod_{i=1}^{m} \left(1 + O\left(2^{1-i}\right)\right)
\right)
\\
&=
\frac{2^{2^{m+2}-3}}{\pi^m}
\;
\left(\frac{2^m-1}{2^m}\right)
\,
\left(
\prod_{i=1}^{m} \left(1 + O\left(2^{1-i}\right)\right)
\right)
\\
&=
\frac{2^{2^{m+2}-3}}{\pi^m}
\;
\left(1 - 2^{-m}\right)
\,
\left(
\prod_{i=1}^{m} \left(1 + O\left(2^{1-i}\right)\right)
\right)\text{.}
\\
\intertext{But
$\prod_{i=1}^{m} \left(1 + O\left(2^{1-i}\right)\right) = O(1)$ and hence}
\prod_{i=1}^{m} \alpha(2^i)
&=
\frac{2^{2^{m+2}-3}}{\pi^m}
\;
O\left(1 - 2^{-m}\right)
\\
&=
O\left(
\frac{2^{2^{m+2}-3}}{\pi^m}
\right)
\text{.}
\end{align*}
\qed
\end{proof}

Having disposed of the preliminary steps,
write $\beta(i) = 2^{m-i} \prod_{j=i}^{m} \alpha(2^j)$ so that our goal
reduces to evaluating $\sum_{i=1}^{m} \beta(i)$.
For $1 \leq i < m$, we first observe that
\begin{align*}
2\;\beta(i+1)
&=
2 \; 2^{m-i-1} \prod_{j=i+1}^{m} \alpha(2^{j}) 
\\
&= 
2^{m-i} \prod_{j=i+1}^{m} \alpha(2^{j}) 
\\
&\leq 
2^{m-i} \prod_{j=i}^{m} \alpha(2^{j}) \qquad\text{(since $\alpha(i) \geq 1$)}
\\
&= 
\beta(i)\text{.}
\\
\intertext{Therefore}
\sum_{i=1}^{m} \beta(i)
&\leq 
\sum_{i=1}^{m} \frac{\beta(1)}{2^{i-1}}
\\
&=
2\,\beta(1) \; \sum_{i=1}^{m} {2^{-i}}
\\
&=
2\beta(1) \, \left(1 - 2^{-m}\right)
\\
&\leq 
2\beta(1)\text{.}
\\
\intertext{Then it follows that}
\sum_{i=1}^{m}
2^{m-i} \prod_{j=i}^{m} \alpha(2^{j})
&\leq 
2\,\beta(1)
\\
&=
2 \; 2^{m-1} \; \prod_{j=1}^{m} \alpha(2^{j})
\\
&=
2^m \; O\left(
\frac{2^{2^{m+2}-3}}{\pi^m}
\right)
\\
&=
O\left(
\frac{2^{2^{m+2}+m-3}}{\pi^m}
\right)\text{.}
\end{align*}
\qed
\end{proof}

% \newpage

% \newpage

% Let $X_{j,j}$ the event that $a_{i,j} = 1$:

% $$
% P[X_{i,j}] =
% \begin{cases}
% p & \text{if $j \leq i$} \\
% 0 & \text{if $j > i$.}
% \end{cases}
% $$

% \medskip
% Let $B=PA$.

% For $1 \leq i \leq n$ and $1 \leq j \leq n$,
% let $\mathcal{X}_{i,j}$ be event $b_{i,j} = 1$.
% We have

% $$
% \PROBABILITY[\mathcal{X}_{i,j}] = \frac{p(n-j+1)}{n}
% $$

% \medskip

% For $1 \leq i \leq n$ and $1 \leq j \leq n$,
% let $\mathcal{Y}_{i,j}$ be the event
% $\sum_{k=1}^{i} b_{k,j} > 0$,
% and let $\INDICATOR_{\mathcal{Y}_{i,j}}$ the indicator variable for event
% $\mathcal{Y}_{i,j}$.

% \begin{align*}
% \PROBABILITY\left[\mathcal{Y}_{i,j}\right]
% &=
% 1 - \PROBABILITY\left[\,\overline{\mathcal{Y}_{i,j}}\,\right] \\
% &=
% 1 - \PROBABILITY\left[ \bigcap_{1\leq k \leq i} \overline{\mathcal{X}_{k,j}}\right] \\
% &=
% 1 - \prod_{k=1}^{i} \left(1 - \PROBABILITY\left[\mathcal{X}_{k,j}\right]\right) \\
% &=
% 1 - \prod_{k=1}^{i} \left( 1 - \frac{p(n-j+1)}{n}\right) \\
% &=
% 1 - \left( 1 - \frac{p(n-j+1)}{n} \right)^i
% \end{align*}

% For $1 \leq i \leq n$, define the random variable
% $Z_i = \sum_{j=1}^{n} \INDICATOR_{\mathcal{Y}_{i,j}}$


% \begin{align*}
% P[Z_i = k]
% &\geq
% \binom{n}{k} \; \left(1-\PROBABILITY\left[\mathcal{Y}_{i,1}\right]\right)^{n-k} \;
% \left(\PROBABILITY\left[\mathcal{Y}_{i,n}\right]\right)^{k} \\
% \intertext{and hence}
% P[Z_i < i+1]
% &\geq
% \sum_{k=0}^{i}
% \binom{n}{k} \;
% \left(1-\PROBABILITY\left[\mathcal{Y}_{i,1}\right]\right)^{n-k} \;
% \left(\PROBABILITY\left[\mathcal{Y}_{i,n}\right]\right)^{k} \\
% &=
% \sum_{k=0}^{i}
% \binom{n}{k} \;
% \left(1 - \left(1 - \left(1 - p\right)^{i}\right)\right)^{n-k}
% \left(1-\left(1 - \frac{p}{n}\right)^{i}\right)^{k} \\
% &=
% \sum_{k=0}^{i}
% \binom{n}{k} \;
% \left(\left(1-p\right)^{i}\right)^{n-k}
% \left(1-\left(1 - \frac{p}{n}\right)^{i}\right)^{k} \\
% &\geq
% \sum_{k=0}^{i}
% \binom{i}{k} \;
% \left(\left(1-p\right)^{i}\right)^{i-k}
% \left(1-\left(1 - \frac{p}{n}\right)^{i}\right)^{k} \\
% &=
% \left(1 + \left(1-p\right)^{i} - \left(1 - \frac{p}{n}\right)^{i}\right)^{i}
% \text{.}
% \end{align*}

% Let $\mathcal{E}(B)$ be the event that $B$ does not have
% the stepwise bounded property.

% \begin{align*}
% \PROBABILITY\left[\mathcal{E}(B)\right]
% &=
% \PROBABILITY\left[\mathcal{E}(B)\right] \\
% &\leq
% \sum_{i=1}^{n} \PROBABILITY\left[Z_i \geq i+1\right] \\
% &=
% \sum_{i=1}^{n} \left(1 - \PROBABILITY\left[Z_i < i+1 \right]\right) \\
% \end{align*}


%%%%%%%%%%%%%%%%%%%%%%%%%%%%%%%%%%%%%%%%%%%%%%%%%%%%%%%%%%%%%%%%%%%%%%%%%%%%%%%

\end{document}

%%%%%%%%%%%%%%%%%%%%%%%%%%%%%%%%%%%%%%%%%%%%%%%%%%%%%%%%%%%%%%%%%%%%%%%%%%%%%%%

